\ktusection{Analizės išvados}

Analizuojant egzistuojančią literatūrą aptikta, kad geri rezultatai klasifikuojant kenksmingas ir nekenksmingas svetaines yra pasiekiami naudojant įvairius metodus: C4.5 sprendimų medį \cite{trees}, atraminių vektorių mašinų \cite{comp}. Šie metodai bus mėginami bandant išrinkti geriausiai tinkantį parinktam panaudos atvejui.

Duomenų rinkiniai yra formuojami jungiant įvairių juodųjų sąrašų duomenis bei naudojant privačius duomenų rinkinius kurie taip pat atitinka juoduosius sąrašus \cite{trees, comp, webcop}.

Tinklalapių parametrai yra formuojami iš jų adresų (URL) \cite{trees, comp}. Interneto grafo struktūra naudojama straipsnyje \cite{linchpins} stengiantis identifikuoti klasterių centrus. Siūloma bandyti šią metriką pritaikyti ir klasifikavimo uždavinyje. Taip pat naudojami papildomi duomenys apie talpinančius serverius, įvairius svetainių metaduomenis, tačiau šie duomenys nėra viešai pasiekiami, ir jų surinkimui naudojami itin dideli skaičiavimo resursai -- \cite{linchpins} straipsnio atveju naudota 20 virtualių mašinų, renkančių duomenis 7 mėnesius. Dėl to naudojami tik URL bei svetainių grafo PageRank įverčiai.

Modeliai formuojami naudojant įvairius metodus, tačiau dalinimas į mokymo, testavimo ir validavimo imtis yra paprastas ir tinkamas, turint pakankamai didelį duomenų rinkinį. Kryžminė validacija gali būti naudojant esant mažesniems rinkiniams \cite{trees}. Modeliai vertinami naudojant įvairias metrikas: tikslumas, validumo vertinimas naudojant AUC, tikslumą, specifiškumą, jautrumą.

\newpage
\ktusection{Darbo tikslas ir uždavinys}
Pavojingi puslapiai skirstomi į skirtingas kategorijas pagal jų vykdomą \cite{tax} ir pagal jų paskirtį žalingų puslapių tinkle \cite{linchpins}. Jų aptikimui naudojami įvairūs metodai \cite{comp}. Šiame darbe bus tobulinamas tokių puslapių aptikimas, naudojant interneto grafo bei puslapių URL informaciją.

Dabro tikslas -- pasiūlyti metodiką ir sukurti įrankį kenksmingų ir nekenksmingų tinklalapių klasifikavimui, kur kenksmingi tinklalapiai yra pavojingų tinklalapių sąrašuose aptinkami adresai, ir įvertinti PageRank įverčio panaudojimą kenksmingų tinklalapių klasifikavimui.

Sprendžiami uždaviniai:
\begin{enumerate}
    \item apžvelgti klasifikavimo metodus naudojamus kenksmingų tinklalapių klasifikavimui;
    \item sukurti programinę įrangą skirtą tinklalapių klasifikavimui į kenksmingus ir nekenksmingus;
    \item įvertinti PageRank įverčio panaudojimą;
    \item pritaikyti metodą suformuotam duomenų rinkiniui.
\end{enumerate}
