\ktusection{Straipsnio Malicious Web Sites Detection using C4.5 Decision Tree apžvalga}

Straipsnyje \kclong{trees} siūlomas sprendimų medžio naudojimas kenkėjiškoms svetainėms identifikuoti pagal įvairias jų
savybes.

\ktusubsection{Problema}

Internetinės svetainės naudoja įvairius pajamų generavimo metodus. Dalis jų yra neteisėti ir kenkia svetainių naudotojams.
Tokių svetainių identifikavimas ir blokavimas leidžia apsaugoti naudotojuos nuo kenkėjiškos programinės įrangos. Įprasti
metodai naudoja pavojingų svetainių sąrašus, tačiau šie sąrašai yra baigtiniai, naujos svetainės į juos pridedamos ne
iš karto. Alternatyva blokuojamų sąrašų sudarymui yra mašininio mokymo modelis kuris gali atpažinti pavojingas svetaines
pagal jų adresą.

\ktusubsection{Metodas}

\ktutexfigure{apzvalga/webtrees_process.tex}{Metodo schema}

Metodo pritaikymo schema yra vaizduojama \vref{fig:apzvalga/webtrees_process.tex} diagramoje. Svetainių klasifikavimui
naudojamas C4.5 sprendimų medis.

Klasifikavimui naudojamos leksikografinės svetainių adreso sąvybės bei serverio savybės. Naudojamos svetainės serverio sąvybės:
\begin{enumerate}[label=\arabic*.]
    \item Duomenų centro vieta
    \item Domeno sąvininko kontaktinė informacija
    \item Domeno registracijos data
    \item Domeno informacijos atnaujinimo data
    \item Svetainės spartinančiųjų atmintinių informacijos saugojimo laikas
    \item Domeno valstybės kodas
    \item Ryšio su serveriu pralaidumas
\end{enumerate}

Naudojamos leksikografinės savybės yra formuojamos iš svetainės universalaus adreso (URL). Jis yra dalinamas į segmentus,
kuriuos skiria įvairūs simboliai bei skyrybos ženklai. Šie segmentai tampa dvireikšmėmis adreso sąvybėmis.

\ktusubsection{Duomenų rinkinio aprašas}

Modeliui apmokyti naudotas duomenų rinkinys surinktas iš kelių skirtingų šaltinių. Naudoti 5000 URL iš kurių 1676 yra
kenkėjiškų svetainių adresai.

\ktusubsection{Rezultatai}
\ktucomment{
jautrumas sensitivity
specifiškumas specificity
tikslumas accuracy
}

Apmokant modelį naudota kryžminė patikra, 10\% duomenų skiriama testavimo duomenų imčiai. Naudoti matai tikslumui
įvertinti yra jautrumas, tikslumas, specifiškumas.

\begin{equation}\label{eq:jautrumas}
jautrumas = {TP \over {TP + FN}} \cdot 100
\end{equation}

\begin{equation}\label{eq:tikslumas}
tikslumas = {TP \over {jautrumas + specifiškumas}} \cdot 100
\end{equation}

\begin{equation}\label{eq:specifiškumas}
specifiškumas = {TN \over {TN + FP}} \cdot 100
\end{equation}

kur \textit{TN} yra tikrų neigiamų, \textit{TP} tikrų teigiamų, \textit{FP} klaidingai teigiamų, \textit{FN} klaidingai
neigiamų klasifikavimų skaičius.

Taip pat skaičiuojama ir ROC kreivė, AUC metrika.

\begin{ktutable}{webtrees_rezultatas}{Straipsnyje pateikiami matavimų rezultatai}
    \begin{tabular}{| l | c | c | c | c | }
     \hline
     \diagbox{Kategorija}{Metrika} & Tikslumas & Jautrumas & Specifiškumas & AUC \\ \hline
     Nekenksmingos & 98.3\% & 96.4\% & 96.5\% & 0.985 \\ \hline
     Kenksmingos & 92.9\% & 96.4\% & 96.5\% & 0.985 \\ \hline
    \end{tabular}
\end{ktutable}

Matavimų rezultatai pateikiami \vref{tab:webtrees_rezultatas}.