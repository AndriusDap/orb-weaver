\ktusection{Straipsnio Malicious Web Sites Detection using C4.5 Decision Tree apžvalga}

Straipsnyje \kclong{trees} siūlomas sprendimų medžio naudojimas kenkėjiškoms svetainėms identifikuoti pagal įvairias jų
savybes.

\ktusubsection{Problema}

Internetinės svetainės naudoja įvairius pajamų generavimo metodus. Dalis jų yra neteisėti ir kenkia svetainių naudotojams.
Tokių svetainių identifikavimas ir blokavimas leidžia apsaugoti naudotojuos nuo kenkėjiškos programinės įrangos. Įprasti
metodai naudoja pavojingų svetainių sąrašus, tačiau šie sąrašai yra baigtiniai, naujos svetainės į juos pridedamos ne
iš karto. Alternatyva blokuojamų sąrašų sudarymui yra mašininio mokymo modelis kuris gali atpažinti pavojingas svetaines
pagal jų adresą.

\ktusubsection{Metodas}

\ktutexfigure{apzvalga/webtrees_process.tex}{Metodo schema}

\ktusubsection{Duomenų rinkinio aprašas}
\ktusubsection{Rezultatai}
\ktusubsection{Programinė įranga}
\ktusubsection{Metodo analizė}