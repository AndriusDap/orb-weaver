Atramos vektorių metodas naudojamas klasifikuoti daugelio matmenų vektorius į dvi klases \cite{comp}. Tai atliekama atskiriant objektus erdvėje naudojant tam tikrą hiperplokštumą. Tai vaizduojama \ktufigref{images/svm.png} paveiksle. Objektų erdvė yra transformuojama naudojant branduolio funkciją, kuri leidžia parinkti tokią erdvę kurioje klasių atskyrimas naudojant hiperplokštumą yra įmanomas.

\ktufigure{images/svm.png}{8 cm}{Supaprastinta atramos vektorių mašinų metodo schema}.

Atramos vektorių mašinos modelis yra kaip \cite{comp}:
\begin{equation}
    h(x) = b + \sum_{n=1}^{N}y_i \alpha_i K(x, x_i)
\end{equation}
kur $h(x)$ yra atstumas nuo hiperplokštumos skiriančios klases, $b$ yra papildomas svorio koeficientas, $y_i$ yra $i$-tojo mokymo rinkinio elemento klasė, o $x_i$ yra $i$-tasis mokymo imties elementas. $N$ yra mokymo imties elementų skaičius, $K$ yra naudojama branduolio funkcija kuri parinkta optimaliai erdvės transformacijai, $x$ yra klasifikuojamas objektas.

Modelis yra formuojamas apibrėžiant \cite{ksvm}:
\begin{equation}
    W(\alpha) = \sum_{i}\alpha_i y_i - \frac{1}{2} \sum_{i, j} \alpha_i \alpha_j K(x_i, x_j)
\end{equation}
ir sprendažitn atramos vektorių mašinų kvadratinio optimizavimo uždavinį:
\begin{equation}
\max_{\alpha} W(\alpha) \text{kur} \left\{
                  \begin{array}{l}
                    \sum_{i} \alpha_i = 0 \\
                    A_i \leq \alpha_i \leq B_i \\
                    A_i = \min(0, Cy_i) \\
                    B_i = \max(0, Cy_i) \\
                  \end{array}
                \right.
\end{equation}
Kur C yra hiperparametras naudojamas modelio formavime. Naudojant mažas $C$ vertes galima pritaikyti modelį triukšmingiems pradiniams duomenims, o didelės vertės galimos kai visi mokymo duomenų rinkinio pavyzdžiai turi teisingas klases.

Šio modelio naudojama branduolio funkcija gali būti laikoma vienu iš hiperparametrų kurie yra derinami modelio mokymo metu. Nagrinėjami trys branduolio tipai \cite{vw}:
\begin{enumerate}
    \item tiesinis
    \item polinominis
    \item radialinės bazės
\end{enumerate}

Tiesinis branduolio tipas yra naudojamas kai funkcija $K$ yra nenaudojama, tokiu atveju atramos vektorių mašinos yra apibrėžiamos kaip
\begin{equation}
    W(\alpha) = \sum_{i}\alpha_i y_i - \frac{1}{2} \sum_{i, j} \alpha_i \alpha_j x_i x_j
\end{equation}.
Šis modelio variantas gali būti naudojamas kai objektai nagrinėjamoje erdvėje gali būti atskirti tiesia plokštumas, nėra reikalingos papildomos modifikacijos.

Polinominis branduolis gali būti formuojamas naudojant \cite{vw}
\begin{equation}
    K(u, v) = (1 + u \cdot v)^d
\end{equation}
radialinės bazės branduolis išreiškiamas
\begin{equation}
    K(u, v) = exp \big( - \frac{|| u - v ||^2}{2 \sigma ^ 2} \big)
\end{equation}

Šis modelis turi du derinamus hiperparametrus - tai yra konstanta $C$, bei parenkama branduolio funkcija. Branduolio funkcija taip pat prideda papildomus hiperparametrus. Polinominio branduolio atveju tai yra laipsnio parametras $d$, radialinės bazės branduolio atveju tai yra parametras $\sigma$.