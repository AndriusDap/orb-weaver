Kuriama metodika tinklalapių klasifikavimui. Siekiamas galutinis rezultatas yra įrankis, kurio įvestis yra tinklalapio adresas, o rezultatas -- klasifikacija su jai priskirtu tikslumu. Tinklapio adresas (URL) yra pakankamas identifikuoti tinklalapius. Taip pat naudojantis juo galima pasiekti patį tinklapį, taip įrankiui paliekama galimybė pridėti papildomas sąvybes, tokias kaip tinklalapio turinio peržiūra.

Klasifikavimo modeliai savo bendroje formoje kaip įvestį naudoja parametrų vektorių, dažniausiai tai yra $n$ ilgio vektorius $X = (x_1, x_2, ... , x_n)$ žymintis klasifikuojamo objekto sąvybes. Sąvybės išreiškiamos skaitinėmis vertėmis. Klasifikatoriaus tikslas - kiekvienam vektoriui $X$ priskirti klasę $Y$. Dėl to vienas pirmų užduočių kuriuos reikia atlikti yra nagrinėjamo objekto, tinklalapio adreso, transformavimas į sąvybių vektorių $X$.

\ktusection{Sąvybių vektoriaus $X$ formavimas}

Tinklalapio sąvybių vektorius $X$ gaunamas iš tinklalapio adreso bei svetainės PageRank įverčių. Tinklalapio adresas yra simbolių eilutė, naiviu atveju būtų galima jį traktuoti kaip skaičių masyvą, tačiau taip būtų prarandama daug informacijos. Visą tinklapio adresą atitinkantis vektorius $X$ yra sudarytas iš sąvybių:
\begin{enumerate}
\item srities varde esantys žodžiai;
\item užklausoje esantys žodžiai;
\item taško simbolių skaičius adrese;
\item specialiųjų simbolių skaičius adrese (Naudojami simboliai:
\texttt{-},
\texttt{.},
\texttt{\_},
\texttt{\~},
\texttt{:},
\texttt{/},
\texttt{?},
\texttt{\#},
\texttt{[},
\texttt{]},
\texttt{@},
\texttt{!},
\texttt{\$},
\texttt{\&},
\texttt{'},
\texttt{(},
\texttt{)},
\texttt{*},
\texttt{+},
\texttt{,},
\texttt{;},
\texttt{=}
ir tarpo simbolis);
\item kenksmingų svetainių PageRank įvertis;
\item nekenksmingų svetainių PageRank įvertis.
\end{enumerate}.
Dėl žodžių naudojimo sąvybių vektoriuje jo ilgis priklauso nuo naudojamo duomenų rinkinio. Kiekvienam žodžiui rinkinyje naudojama po vieną lauką vektoriuje, dėl to jis gali tapti ganėtinai ilgas.

\ktufigure{images/url_struktura.png}{15 cm}{Tinklalapio adreso (URL) struktūra}

Pavyzdinis tinklalapio adresas yra pateikiamas \ktufigref{images/url_struktura.png} paveiksle. Čia matoma ir jo struktūra. Adreso pradžioje yra protokolas, dažniausiai aptinkami \texttt{http://} ar \texttt{https://}. Antras segmentas yra sritis. Kreipiantis į tinklalapį naudojamas sričių vardų serveris, kuris susieja simbolinį adresą, sritį, su skaitiniu adresu, pagal kurį galima kreiptis į svetainę talpinantį serverį. Sritis parenkama taip, kad atspindėtų svetainės turinį, būtų lengvai įsimenama naudotojams, bei būtų lengvai randama paieškos varikliuose. Sričių vardus suteikia sričių registrai, už tam tikrą mokestį. Kaina nustatoma pagal raktažodžių populiarumą. Paskutinis adreso segmentas yra užklausa. Ji leidžia svetainė talpinančiam serveriui pateikti tą dokumentą kuris atitinka tinklalapį kurį bando pasiekti naudotojas.

Išskiriamos sąvybės iš tinklalapio adreso yra:
\begin{enumerate}
\item srities varde esantys žodžiai;
\item užklausoje esantys žodžiai;
\item taško simbolių skaičius adrese;
\item specialiųjų simbolių skaičius adrese.
\end{enumerate}
Žodžiai adrese yra išskiriami kaip dviejų reikšmių sąvybės, $0$ jei žodis neegzistuoja adrese, $1$ jei žodis yra adrese. Laikoma, kad žodžius skiria specialieji simboliai.

Pagerank įverčiai gaunami iš kryptinio svetainių grafo $G = (V, E)$, kur $V$ yra šio grafo viršūnės o $E$ lankai. Grafo formavimas iš interneto svetainių yra pavaizduotas \ktufigref{images/hostgraph.png} paveiksle.
\ktufigure{images/hostgraph.png}{15 cm}{Svetainių grafo formavimas iš svetainių}

Grafo viršūnės $V$ yra svetainės, o lankai atitinka ryšius tarp svetainių. Tai yra nuorodos tarp svetainių aptinkamos svetainių tinklalapiuose. Šie lankai turi kryptį. Jei svetainės $v_a$ tinklalapyje yra nuoroda į svetainės $v_b$ tinklalapį laikoma, kad egzistuoja lankas $e_{ab}$ iš viršūnės $v_a$ į viršūnę $v_b$. Visų lankų svoris yra vienodas, tai yra jei keli svetainės $v_a$ tinklalapiai turi nuorodas į svetainės $v_b$ tinklalapius, visvien laikoma kad yra tik vienas lankas $e_{ab}$. Galimi abipusiai svetainių ryšiai, tai yra jei egzistuoja lankas $e_{ab}$ lankas $e_{ba}$, kuris rodo ryšį iš viršūnės $v_b$ į viršūnę $v_a$ taip pat gali egzistuoti.


