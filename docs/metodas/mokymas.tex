\ktusection{Kenksmingų interneto tinklalapių identifikavimo modelio apmokymas}

Kenksmingų interneto tinklalapių identifikavim modelio apmokymo schema yra pateikiama paveiksle \ktufigref{images/mokymas.png}.
\ktufigure{images/mokymas.png}{13 cm}{Modelio formavimo schema}

Modeliams mokyti naudojamas duomenų rinkinys yra dalinamas į tris imtis:
\begin{enumerate}
    \item mokymo imtis (70\% viso rinkinio);
    \item testavimo imtis (25\% viso rinkinio);
    \item validavimo imtis (5\% viso rinkinio).
\end{enumerate}
Didžiausia, mokymo imtis yra naudojama modelio apmokymui. Siekiant aptikti modelių persimokymą naudojama
testavimo imtis. Modelį apmokius yra įvertinama jo kokybė naudojant parinktas metrikas, jei įverčiai stipriai
skiriasi mokymosi ir testavimo imčiai modelis yra persimokęs, dėl to reikia koreguoti modelio hiperparametrus.
Jei gauta modelio kokybė netenkina,  taip pat koreguojami hiperparametrai.

Suformavus modelį kuris tenkina jis yra tikrinamas paskutinį kartą. Tam naudojama validavimo imtis.
Modelio rezultatai naudojami įvertinti modelio klasifikavimo kokybę.


Modelio klasifikavimo kokybės įverčiai yra tikslumas, AUC įvertis, specifiškumas, jautrumas. Šios metrikos leidžia
 įvertinti modelį net ir esant klasių disbalansui, ko vien naivus teisingų spėjimų procentas parodyti
  negali. Klasifikavimo rezultatų lentelę sudaro keturis rezultatų tipai pagal jų teisingumą:
\begin{enumerate}
    \item teisingai teigiamą rezultatą (žymima $TP$);
    \item klaidingai teigiamą rezultatą (žymima $FP$);
    \item teisingai neigiamą rezultatą (žymima $TN$);
    \item klaidingai neigiamą rezultatą (žymima $FN$).
\end{enumerate}

 Atsižvelgiant į ROC kreivę
(angl. \textit{receiver operating characteristic curve}), galima nustatyti
klasių intervalus kurie yra optimalūs modelio naudojimui. Ši kreivė rodo $TP$ tipo įverčio
tikimybės santykį su $FP$ tipo įverčio tikimybe keičiant ribinę klasifikatoriaus vertę, pagal
kurią yra parenkama klasė. AUC metrika yra skaičiuojama kaip plotas, esantis po šia kreive.
Kitos klasifikavimo kokybės metrikos apskaičiuojamos:

\begin{equation}
\text{jautrumas} = {TP \over {TP + FN}}
\end{equation}

\begin{equation}
\text{tikslumas} = {TP \over {\text{jautrumas} + \text{specifiškumas} }}
\end{equation}

\begin{equation}
\text{specifiškumas} = {TN \over {TN + FP}}
\end{equation}

Šių kokybės metrikų galimos reikšmės yra intervale $[0, 1]$, siekiama, kad jos būtų kuo arčiau $1$.
