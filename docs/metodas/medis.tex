C4.5 sprendimų medis yra sprendimų medžio tipo klasifikatorius kurio formavimui naudojamas C4.5 algoritmas.
Sprendimų medis yra programa, suformuota iš tam tikrų testų, kurie tikrina nagrinėjamo objekto parametrus,
ir pagal juos parenkama tam tikra klasė.

\ktufigure{images/sprendimu_medis.png}{8 cm}{Sprendimų medžio pavyzdys}

Pavyzdinis sprendimų medis klasifikuojantis vektorių $X$ į klases $Y = (y_1, y_2, y_3)$ pateikiamas
 \ktufigref{images/sprendimu_medis.png} paveiksle. Jį sudaro testų rinkinys $T$.

Sprendimų medį sudarantys testai yra formuojami modelio apmokymo metu. Mokymui naudojamas duomenų
rinkinys $H$ kurį sudaro mokymosi duomenys su jiems priskirtomis klasėmis. Medžio formavimo metu
parenkami testai kurie maksimizuoja funkciją:
\begin{equation} \label{eq:gainratio}
    santykinisIšlošis(t_i) = \frac{išlošis(t_i)}{santykinėEntropija(t_i)}
\end{equation}
Santykinis išlošis priklauso nuo informacijos išlošio, kuris būtų maksimizuojamas naudojant ID3
algoritmą bei santykinės entropijos. Santykinė entropija naudojama siekiant sumažinti informacijos
išlošio iškreipimą kai testas skaido rinkinį į daug skirtingų klasių \cite{c45}
\begin{equation}
    išlošis(t_i) = entropija(H) - entropija_{t_i}(H)
\end{equation}
informacijos išlošis vertina duomenų rinkinio klasių atsiskyrimą - skaičiuojama kiek informacijos
suteikia padalinimas naudojant nagrinėjamą testą

\begin{equation}
    entropija_{t_i} = \sum_{j=1}^{n} \frac{|H_i|}{H} \times entropija(H_i)
\end{equation}

\begin{equation}
    entropija(S) = - \sum_{j=1}^{k} \frac{freq(C_j, S)}{|S|} \times log_2 \big( \frac{freq(C_j, S)}{|S|} \big)
\end{equation}

\begin{equation}
    santykinėEntropija(S) =  \sum_{j=1}^{n} \frac{|H_i|}{H} \times log_2 \big( \frac{|H_i|}{H} \big)
\end{equation}

kur $freq(C_j, S)$ žymi klasės $C_j$ pasikartojimo kiekį rinkinyje $S$.
Aptikus testą kuris maksimizuoja santykinį išlošį \ref{eq:gainratio} testas yra įtraukiamas į medį.
Mokymo duomenų rinkinys $H$ yra skaidomas pagal testą į kelis rinkinius\cite{c45}. Šie duomenų
rinkiniai priskiriami medžio šakoms atitinkančioms testo rezultatus ir rekursiškai tęsiamas medžio formavimas.

C4.5 medžio formavimo algoritmas palaiko diskrečių reikšmių parametrus bei tolygius kintamuosius.
Testai vykdomi diskretiems parametrams sukuria po vieną šaką kiekvienai galimai parametro reikšmei.

Testai tolydiems parametrams vykdomi parenkant ribinę vertę $Z$, su kuria bus lyginama parametro
$A$ vertė \cite{c45}. Kadangi apmokymui naudojamas baigtinis duomenų rinkinys, jame yra baigtinis
kiekis galimų tolydaus parametro $A$ reikšmių $\{v_1, v_2, ..., v_m\}$. Ribinei vertei $Z$ priskiriama
viena iš egzistuojančių parametro $A$ verčių  $v_i$. Parenkama vertė kuri maksimizuoja
santykinį išlošį \ref{eq:gainratio}.

Suformavus medį vykdomas medžio mažinimas. Dalis šakų yra sukeliančios persimokymą - tai yra
jos yra per daug pritaikytos mokymo imčiai \cite{c45}. Šios šakos yra aptinkamos skaičiuojant
tikėtiną klaidų dažnį medžiui be tam tikro testo ir su potencialiai pertekliniu testu.

Šis metodas neturi hiperparametrų kuriuos būtų galima derinti.