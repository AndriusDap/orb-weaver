\newcommand{\ktusubject}{Baigiamasis magistro projektas}
\newcommand{\ktutitle}{Kenksmingų interneto tinklalapių identifikavimo modelis}
\newcommand{\ktuyear}{2017}
\newcommand{\ktudate}{\ktuyear-06-02}
\newcommand{\ktucity}{Kaunas}
\newcommand{\ktuinstitution}{Kauno technologijos universitetas}
\newcommand{\ktufaculty}{Matematikos ir gamtos mokslų fakultetas}
\newcommand{\ktucathedral}{Taikomosios matematikos katedra}
\newcommand{\ktuauthor}{Andrius Dapševičius}
\newcommand{\ktumastera}{doc. dr. Vytautas Janilionis}
\newcommand{\ktumasterb}{prof. dr. Rimantas Gatautis}


\newcommand{\kclong}[1]{\cite{#1}}
\newcommand{\kc}[1]{\citeauthor{#1} \cite{#1}}
\newcommand{\itrain}{$\mathbf{T_m}$}
\newcommand{\train}{\mathbf{T_m}}

\documentclass[12pt, a4paper, onecolumn, titlepage, oneside, intlimits]{report}

\usepackage{bm}

\usepackage{ktua4}

\setmainfont[
    Path           = ./fonts/,
    Extension      = .TTF,
    Ligatures      = TeX,
    BoldFont       = *bd,
    ItalicFont     = *i,
    BoldItalicFont = *bi
]{Times}
\setmonofont[
    Path           = ./fonts/,
    Extension      = .ttf,
    Ligatures      = TeX,
    BoldFont       = *bd,
    ItalicFont     = *i,
    BoldItalicFont = *bi
]{cour}

\usepackage{commath}

\usepackage{unicode-math}

\usepackage{diagbox}
\usepackage{polyglossia}
\setdefaultlanguage{lithuanian}

\usepackage{tikz}
\usetikzlibrary{shapes.geometric, arrows}

\usepackage[spacecolon=false,bibencoding=UTF8,sortlocale=en_US,backend=biber,sorting=none,labelnumber=true,natbib=true,bibstyle=iso-numeric,firstinits=true,citestyle=iso-numeric,language=lithuanian]{biblatex}
\bibliography{references}

\begin{document}

\ktuinit{}
\includepdf{santrauka.pdf}
\newpage

\ktuchapter{Įžanga}

Šiuolaikinės technologijos lemia spartų interneto tinklapių atsiradimą. Tinklalapiai kuriami siekiant gauti kokią nors naudą. Tačiau dalis jų yra kuriama tikintis pasipelnyti naudotojams kenksmingais būdais, tai yra infekuojant jų kompiuterius, vagiant asmens duomenis. Tokie tinklalapiai yra kenksmingi ir juos atpažinti yra svarbu, taip apsaugant naudotojus.

Darbo tikslas - pasiūlyti ir įgyvendinti metodiką tinklalapių klasifikavimui į kenksmingus bei nekenksmingus pasinaudojant jų adresu bei svetainių grafo suteikiama informacija. Tai leidžia tiksliau klasifikuoti kenksmingus tinklalapius remiantis prielaida, kad kenksmingi tinklalapiai svetainių grafe yra susiję, tai yra kenksmingi tinklalapiai turi nuorodas į kenksmingus tinklalapius, nekenksmingi tinklalapiai turi daugiau nuorodų į nekenksmingus tinklalapius.

Pirmoje darbo dalyje yra pateikiama klasifikavimo metodų bei kenksmingų tinklalapių tyrinėjimo literatūros apžvalga. Antroje dalyje pateikiama metodika kenksmingų ir nekenksmingų tinklalapių klasifikavimui, suformuojant mašininio mokymo modelius. Trečiojoje dalyje yra aprašomas ir apmokomas modelis, pateikiama programinė įranga šio modelio praktiniam panaudojimui. Taip pat trečioje dalyje kritiškai įvertinamas praktinis įrankio panaudojimas.

\newpage

\ktuchapter{Literatūros apžvalga}

Internetas yra vienas iš naujų reiškinių atsiradusių pasaulyje. Jis gali būti nagrinėjamas įvairiomis perspektyvomis. Jis gali atlikti informacijos šaltinio paskirtį, leisti praleisti laisvalaikį socialiniuose tinkluose, apsipirkinėti įvairiose parduotuvėse neišeinant iš namų. \ktufigure{images/internetas.png}{10 cm}{Interneto svetainių ir tinklapių struktūra} Tačiau žvelgiant iš techninės pusės, internetas yra daugybė jame pasiekiamų svetainių. Svetaines sudaro vienas ar daugiau puslapių, dar vadinamų tinklalapiais \ktufigref{images/internetas.png}. Šie tinklalapiai talpina tam tikrą turinį bei nuorodas į kitus tinklalapius, taip suformuojant tinklą -- kryptingą grafą kurį sudaro tinklalapiai bei naudotojams naviguoti skirtos nuorodos vedančios į kitus tinklalapius.

Svetainėms kurti ir talpinti yra reikalingi žmogiškieji resursai turinio sukūrimui, programinės įrangos paruošimui. Taip pat ir infrastruktūra jų talpinimui. Tai sudaro jų talpinimo kaštus, kuriuos padengia svetainė turi tiesiogiai ar netiesiogiai padengti, nesvarbu ar tai yra e-parduotuvė ar įmonę reprezentuojantis puslapis skirtas pritraukti klientus. Šie pajamų generavimo metodai yra paremti naudos suteikimu. Naudojami metodai gali būti tiesioginiai - apmokestinami parduodami produktai ar prieiga prie turinio, ar reklama paremtas monetizacijos modelis -- tinklalapiuose talpinamos nuorodos į kitas svetaines, siūlomi produktai. Šie metodai yra naudingi vartotojams, jie gauna informaciją, sužino apie naujus produktus, savo noru perka prekes.

Dalis svetainių yra paremtos naudotojams žalingais monetizacijos modeliais \cite{tax}, kuriais siekiama pasinaudojant vartotojo naivumu, techniniu neišprusimu ar jo apgaule išviliojami asmeniniai duomenys, įdiegiama kenkėjiška programinė įranga į jo kompiuterį.
Straipsnyje \kclong{tax} analizuojami įvairūs metodai kuriais pasinaudojant apgaunami naudotojai, jie patenka į kenksmingus tinklalapius. Atakose naudotojas yra išskiriamas kaip silpniausia sistemos grandis \cite{tax}. Tai išskiriama ne tik dėl galimybės apgauti pateikiant klaidinančią informaciją, bet ir naudotojų turimų teisių, naudotojas gali prieiti prie visų sistemos resursų ir įdiegti bet kokią programinę įrangą, net jei ji yra kenksminga. Sistemos naudotoju stengiamasi pasinaudoti semantinėse atakose.

Straipsnyje išskiriamos semantinės atakos yra \cite{tax}:
\begin{enumerate}
    \item \label{phishing} \textbf{Phishing} -- Sukčiavimas apsimetant, naudojami el. laiškai, žinutės, kuriomis apsimetama patikimu šaltiniu ir siekiama apgaule išvilioti naudotojo asmeninius duomenis, kuriais pasinaudojus bus galima pasipelnyti.
    \item \textbf{File Masquerading} -- Failo maskavimas, platinami failai kurie yra maskuojami kaip naudotojui įprasti ir saugiai atrodantys failai. Failai galimai maskuojami kaip dokumentų failai, sisteminiai failai, kuriuos naudotojas įpratęs matyti ir pasitikėti.
    \item \textbf{Application Masquerading} -- Apsimetimas aplikacija, apsimetama vartotojui naudinga aplikacija.
    \item \label{popup} \textbf{Web Pop-Up} -- Iššokantieji langai, apsimetantys klaidų pranešimais, klausimynais, ar informacija kuri gali sudominti naudotoją.
    \item \label{malvertisement}\textbf{Malvertisement} -- Kenksmingos reklamos.
    \item \label{socnet}\textbf{Social Networking} -- socialiniais tinklais plintančios atakos, tai gali būti pasidalintos nuorodos, failai.
    \item \textbf{Removable Media} -- Fiziniais įrenginiais paremtos atakos.
    \item \textbf{Wireless} -- belaidžiais įrenginiais paremtos atakos.
\end{enumerate}
Didelė dalis šių atakų (\ref{phishing}, \ref{popup}, \ref{malvertisement}, \ref{socnet}) naudojasi tarpinėmis svetainėmis. Tai yra ramstinės svetainės per kurias yra nukreipiamas duomenų srautas ir paskirstomas į kitas kenksmingas svetaines kurios yra naudojamos galutinei naudotojų monetizacijai \cite{linchpins}. Šios atakos pasižymi tuo, kad naudotojai jas gali atskirti per URL adresą, kuris nuves į kenkėjišką svetainę \cite{tax}.

Semantines atakas atpažinti siūlomi mašininio mokymosi modeliai \cite{tax}. Jie gali būti pritaikyti atpažinti atakas pagal įvairias savybes: el. laiškų turinį, svetainių URL adresus, tinklalapių turinį \cite{tax}.
Straipsnyje \kclong{webcop} siūlomas \mq{WebCop} metodas skirtas atpažinti žalingus internetinius tinklalapius ir kenksmingų tinklapių grupes, kaimynystes.

Šiuo straipnsiu siekiama \cite{webcop}:
    \begin{itemize}
        \item pateikti sistemą gebančią atpažinti kenksmingas bei saugias interneto kaimynystes kurias sudaro svetainių nuorodos
         \item pateikti tikslingą, iš apačios aukštyn, veikiantį kenkėjiškos programinės įrangos aptikimo metodą
         \item papildomą būdą atpažinti pirmojo tipo klaidas kenkėjiškos programinės įrangos identifikavimo modulyje
         \item pateikti naują būda atrasti kenkėjišką programinę įrangą.
    \end{itemize}

Straipsnyje aprašoma žalingos programinės įrangos plitimo problema. Vienas iš būdų kaip į kompiuterius patenka  tokia programinė įranga yra kenksmingų vykdomųjų failų parsisiuntimas iš interneto svetainių. Kenkimo programų plitimą siekiama apriboti naudojant prevencines priemones. Siekiama identifikuoti svetaines platinančias kenksmingus failus ir neleisti naudotojams jas pasiekti, ar juos kitaip perspėti apie potencialiai kenksmingą svetainę.

Straipsnyje identifikuotą kenkėjiškos programinės įrangos plitimo problemos sprendimui siūloma WebCop sistema. Tai yra sistema, atpažįstanti kenkėjiškas ir joms gretimas svetaines. Pavyzdžiui jei turima struktūra atitinkanti paveiklso \ktufigref{images/kenksmingos_A_B.png} tinklalapius. Šiuo atveju tinklalapiai A ir B nuorodas turi tik į kenksmingą programinę įrangą, dėl to galima laikyti, kad vartotojas negaus jokios naudos iš jų, tik bus skatinamas įsidiegti kenksmingą programinę įrangą. Taip pat, keliaujant interneto grafu nuo apačios, pradedant kenksminga programine įranga galima aptikit ją platinančias svetaines, ir jas taip pat pažymėti kaip kenksmingas.

\ktufigure{images/kenksmingos_A_B.png}{10 cm}{Kenksmingos programinės įrangos platinimo svetainės struktūros pavyzdys}

WebCop sistemą sudaro WebCop modulis bei dvi duomenų bazės. Viena duomenų bazėje saugomi interneto svetainių tarpusavio ryšių informacija, tai yra interneto svetainių grafas. Antra duomenų bazė laiko informaciją apie identifikuotas kenkėjiškas programas bei jų šaltinius internete. Ši sistema veikia keliais etapais:
    \begin{enumerate}[label=\arabic*.]
        \item nuorodų į platinimo svetaines aptikimas;
        \item žalingų kaimynysčių aptikimas;
        \item naujos, potencialiai žalingos, programinės įrangos aptikimas.
    \end{enumerate}

Pirmas žingsnis, platinimo svetainių atpažinimas, vykdomas jungiant identifikuotų kenkėjų duomenų bazės duomenis su interneto grafo viršūnėmis. Sekančiame žingsnyje išrenkamos svetainės gretimos platinimo svetainėms. Tai yra vienos nuorodos atstumu nuo kenksmingų svetainių esančios svetainės. Šie interneto tinklalapiai yra laikomi žalingos programinės įrangos katalogais. Naudotojų prieiga prie šių tinklalapių turi būti apribota, siekiant juos apsaugoti. Svetainės esančios dviejų nuorodų atstumu ir talpinančios nežinomas taikomąsias programas identifikuojamos kaip potencialiai kenksmingos, laikoma kad tai nauja, neidentifikuota kenkėjiška programinė įranga.

WebCop sistema naudoja du duomenų rinkinius -- kenksmingos programinės įrangos registrą bei interneto grafą. Interneto grafo duomenų rinkinys formuotas naudojant interneto paieškos variklio duomenis. Kenkėjišių programų registras kuriamas naudojantis Windows operacinės sistemos naudotojų teikiamais duomenimis.

\begin{ktutable}{webcop_rezultatas}{WebCop straipsnyje pateikiami matavimų rezultatai}
    \begin{tabular}{| l | r |}
     \hline
        Matavimas & Vertė \\ \hline
        Aptiktos nežalingų programų platinimo svetainės & 1460 \\ \hline
        Aptiktos žalingų programų platinimo svetainės & 10853 \\ \hline
        Aptikti nežalingų programų katalogai & 2850883 \\ \hline
        Aptikti žalingų programų katalogai & 391893 \\ \hline
    \end{tabular}
\end{ktutable}

Lentelėje \vref{tab:webcop_rezultatas} pateikiami straipsnyje atliktų matavimų rezultatai.

Šis metodas yra unikalus keliais aspektais -- skenavimas iš viršaus žemyn pakeičiamas skenavimu iš apačios žemyn ir naudojamas egzistuojantis kenkėjiškos programinės įrangos registras vietoje bandymo ją identifikuoti skenavimo metu. Tai leidžia pasinaudoti interneto grafo struktūra, ne tik atskirų svetainių bei tinklalapių savybėmis.
Interneto grafo savybėmis naudojamasi ir straipsnyje \kclong{linchpins}. Čia aprašomas metodas kenksmingos programinės įrangos aptikimui remiantis interneto grafo topologine struktūra. Interneto svetainės yra ranguojamos naudojant PageRnka algoritmą. Taip aptinkamos svetainės kurios yra svarbios kenksmingų svetainių struktūroje.

Kertinės svetainės straipsnyje \cite{linchpins} yra apibrėžiami kaip dedikuoti serveriai kurie yra būtini kenksmingų internetinių svetainių veikimui. Šie serveriai talpina svetaines, kurios pasiekiamos tik iš kenkėjiškų svetainių ir yra skirti valdyti įvairias kenksmingas veiklas nukreipiant naudotojų srautus. Šiuos tinklo ramsčius aptikti ir sustabdyti yra svarbu dėl vykdomo veiklos masto -- jie gyvuoja daug ilgiau nei įprastos kenksmingos svetainės, apdorojami dideli duomenų kiekiai, dėl to jų sustabdymas labiau sutrikdo žalingą veiklą. Sprendžiama šių ramsčių aptikimo problema.

Svetainės kurias stengiamasi aptikti pasižymi tam tikromis topologinėmis savybėmis grafe. Interneto svetaines galima skaidyti į dvi klases - kenksmingos ir nekenksmingos svetainės. Ramstinių serverių tikslas yra nukreipti naudotojus į kitas kenksmingas svetaines, kur bus bandoma monetizuoti naudotojus. Tai vaizduojama \ktufigref{images/linchpin.png}. Čia tinklapiai A, B, C yra kenksmingi tinklapiai kurie yra skirti pritraukti naudotojų dėmesį. Jie gali būti įterpiami į nekenksmingas svetaines įsilaužus ar naudojant apgaulingas reklamas \cite{tax}. Naudotojai iš jų yra nukreipiami į tarpinį, srauto paskirstymo tinklapį. Šis tinklapis vėl naudotojus automatiškai nukreipia į tolimesnius puslapius, kur bus bandoma iš jų pasipelnyti, ar verčiant ką nors nusipirkti, bandant išvilioti asmeninius duomenis ar kita. Ši ramstinių serverių paskirtis, srauto paskirstymas lemia tai, kad ramstiniai serveriai neturi jokių tiesioginių ryšių iš nekenksmingų svetainių. Visi ryšiai ateina tik iš kenksmingų svetainių. \cite{linchpins}. Tokia topologinė struktūra interneto grafe leidžia juos ranguoti naudojant PageRank algoritmą ir tikėtis aukšto rango.

\ktufigure{images/linchpin.png}{10 cm}{Vartotojų srauto paskirstymo schema}

PageRank algoritmas ranguoja interneto puslapius pagal kiekvieno puslapio svarbą. Svarba vertinama pagal nuorodų į puslapį kiekį, laikoma kad svarbesni puslapiai yra dažniau minimi kitose svetainėse. Taip pat svarbių svetainių paminėjimai yra verti daugiau nei nesvarbių svetainių. Taip imituojamas socialinio tinklo prestižo statusas \cite{Wu2008}.

Vertinant svetainės PageRank įvertį internetas laikomas orientuotu grafu $G = (V, E)$ kur $V$ yra grafo viršūnės, tai yra puslapiai, o $E$ yra orientuotos briaunos, tai yra nuorodos tarp puslapių \cite{Wu2008}. Supaprastintas PageRank įvertis yra apibrėžiamas kaip \cite{pagerank}:
\begin{equation}
    R(u) = c \sum_{v \in B_u} \frac{R(v)}{N_v}
\end{equation}
kur $R(u)$ yra PageRank įvertis svetainei $u$, $B_u$ yra aibė svetainių kurios turi nuorodas į svetainę $u$, $N_v$ yra svetainės $v$ nuorodų į kitus puslapius skaičius, o $c$ normalizavimo faktorius. Šis įvertis yra teisingas, tačiau gali būti lengvai iškreipiamas grafe esančių ciklų. Jie lemtų greitą kilimą range. Tas sprendžiama įtraukiant slopinimo faktorių $d$ \cite{pagerank}. Gaunamas galutinis PageRank įvertis yra:
\begin{equation}
    R(u) = (1 - d) + d \sum_{v \in B_u} \frac{R(v)}{N_v}
\end{equation}
slopinimo faktorius $d$ gali būti nustatomas bet kokiam skaičiui tarp $0$ ir $1$. $d=0.85$ naudojama straipsnyje \cite{pagerank}

 PageRank įvertis priklauso nuo kitų svetainių įverčio, dėl to naudojamas iteratyvinis metodas jo skaičiavimui. Rezultatai taip pat priklauso nuo pradinio, nulinės iteracijos, PageRank vertės nustatymo. Ji gali būti parenkama pagal ekspertinę nuomonę, jei siekiama gauti įprastinį rangavimą kuris tinkamas naudoti interneto paieškos varikliuose.  Iteratyvus algoritmas konverguoja greitai, 322 milijonų nuorodų duomenų rinkinys pasiekia gerą rezultatą per 52 iteracijas \cite{pagerank}.

 PageRank algoritmo rezultatas labai priklauso nuo nulinės iteracijos PageRank įverčių parinkimo \cite{linchpins}. Tuo pasinaudojant formuojami keli PageRank įverčiai. Vienas yra įprastinis įvertis, kai įvertis nustatomas į 1 nekenksmingoms svetainėms, į 0 visoms kitoms svetainėms \cite{linchpins}. Taip suranguojamos nekenksmingos svetainės. Šiuo atveju ramstinės kenksmingos svetainės turės itin žemą rangą, nes neegzistuos nuorodos iį nekenksmingų svetainių. Formuojant antrą PageRank įvertį vieneto reikšmės priskiriamos kenksmingoms svetainėms. Taip ramstinės svetainės gaus itin aukštą rangą \cite{linchpins}. Taip gaunamas rangas interneto tinkle ir kenksmingų svetainių tinkle. Skirtumas tarp rangų leidžia atpažinti ramstines svetaines, taip netiesiogiai pasinaudojant jų izoliuotumu nuo nekenksmingų svetainių interneto grafe.

Naudotas duomenų rinkinys suformuotas iš kelių šaltinių. Kenksmingos svetainės surinktos iš Microsoft suteikto duomenų rinkinio, \mq{WarningBird} projekto duomenų, \mq{Twitter} socialinio tinklo duomenų, \mq{Alexa} svetainių katalogo duomenų. Iš šių šaltinių surinkta apie 5.5 milijono svetainių URL, kurie buvo panaudoti tolimesniam interneto svetainių  rinkimui naudojant paiešką internete \cite{linchpins}. Paieška vykdyta 7 mėnesius naudojant 20 virtualių mašinų \cite{linchpins}.

Duomenys kategorizuoti suformuojant svetainių klasterius ir juos peržiūrint. Keli klasteriai identifikuoti kaip nekenksmingos svetainės. Kenksmingos svetainės žymėtos jas tikrinant naudojant Microsoft Forefront kenksmingos programinės įrangos aptikimo įrankį \cite{linchpins}. Didelė dalis URL (78.51\%) nebuvo priskirti jokiai klasei, jiems klasė priskiriama tyrimo metu, identifikuojant kenksmingas svetaines.

Naudojant turimą duomenų rinkinį dalis žinomų ramstinių serverių yra naudojama kitų ramsčių aptikimui. Atsitiktinai parenkama dalis žinomų ramsčių (1\%, 5\%, 10\%, 50\%, 90\%) kurie bus naudojami kaip nulinė PageRank įverčio skaičiavimo iteracija, jiems kenksmingų svetainių formavimo žingsnyje nustatomas pradinis $R(u) = 1$, ir tikrinama kiek kitų žinomų ramsčių bus aptikta.

Naudojant 5\% pradiniam apmokymui aptikta 48.59\% kitų ramstinių serverių. Tai reiškia kad įmanoma aptikti 7 kartus daugiau ramstinių serverių nei buvo panaudota apmokyti pradiniam modeliui. Šiuo atveju pirmojo tipo klaidos pasiekia 2.36\%. Tai yra atvejis pasiekęs didžiausią klaidą ir skirtumą tarp pradinio duomenų rinkinio dydžio ir aptiktų ramstinių serverių kiekio. Metodas gali būti taikomas iteratyviai, panaudojant dalį aptiktų ramstinių serverių kitų serverių radimui. Tai leidžia aptikti žymiai daugiau ramstinių serverių, tačiau klaidos vis dar yra pakankamai žemos, leidžiančios praktinį panaudojimą \cite{linchpins}.

Naudojant šį metodą pasinaudota interneto grafo topologija, naudoti papildomi duomenys - tai yra domeno informacija, svetaines talpinančių serverių duomenys \cite{linchpins}. Šie duomenys yra galimi naudoti dėl turimo kompiuterinių resursų kiekio -- duomenis rinko 7 mėnesius, naudota 20 įrenginių. Norint tai atlikti naudojant vieną kompiuterį prireiktų daugiau nei 10 metų, tai sukelia sunkumų formuojant tokį duomenų rinkinį.
\ktusection{Straipsnio Malicious Web Sites Detection using C4.5 Decision Tree apžvalga}

Straipsnyje \kclong{trees} siūlomas sprendimų medžio naudojimas kenkėjiškoms svetainėms identifikuoti pagal įvairias jų
savybes.

\ktusubsection{Problema}

Internetinės svetainės naudoja įvairius pajamų generavimo metodus. Dalis jų yra neteisėti ir kenkia svetainių naudotojams.
Tokių svetainių identifikavimas ir blokavimas leidžia apsaugoti naudotojuos nuo kenkėjiškos programinės įrangos. Įprasti
metodai naudoja pavojingų svetainių sąrašus, tačiau šie sąrašai yra baigtiniai, naujos svetainės į juos pridedamos ne
iš karto. Alternatyva blokuojamų sąrašų sudarymui yra mašininio mokymo modelis kuris gali atpažinti pavojingas svetaines
pagal jų adresą.

\ktusubsection{Metodas}

\ktutexfigure{apzvalga/webtrees_process.tex}{Metodo schema}

Metodo pritaikymo schema yra vaizduojama \vref{fig:apzvalga/webtrees_process.tex} diagramoje. Svetainių klasifikavimui
naudojamas C4.5 sprendimų medis.

Klasifikavimui naudojamos leksikografinės svetainių adreso sąvybės bei serverio savybės. Naudojamos svetainės serverio sąvybės:
\begin{enumerate}[label=\arabic*.]
    \item Duomenų centro vieta
    \item Domeno sąvininko kontaktinė informacija
    \item Domeno registracijos data
    \item Domeno informacijos atnaujinimo data
    \item Svetainės spartinančiųjų atmintinių informacijos saugojimo laikas
    \item Domeno valstybės kodas
    \item Ryšio su serveriu pralaidumas
\end{enumerate}

Naudojamos leksikografinės savybės yra formuojamos iš svetainės universalaus adreso (URL). Jis yra dalinamas į segmentus,
kuriuos skiria įvairūs simboliai bei skyrybos ženklai. Šie segmentai tampa dvireikšmėmis adreso sąvybėmis.

\ktusubsection{Duomenų rinkinio aprašas}

Modeliui apmokyti naudotas duomenų rinkinys surinktas iš kelių skirtingų šaltinių. Naudoti 5000 URL iš kurių 1676 yra
kenkėjiškų svetainių adresai.

\ktusubsection{Rezultatai}
\ktucomment{
jautrumas sensitivity
specifiškumas specificity
tikslumas accuracy
}

Apmokant modelį naudota kryžminė patikra, 10\% duomenų skiriama testavimo duomenų imčiai. Naudoti matai tikslumui
įvertinti yra jautrumas, tikslumas, specifiškumas.

\begin{equation}\label{eq:jautrumas}
jautrumas = {TP \over {TP + FN}} \cdot 100
\end{equation}

\begin{equation}\label{eq:tikslumas}
tikslumas = {TP \over {jautrumas + specifiškumas}} \cdot 100
\end{equation}

\begin{equation}\label{eq:specifiškumas}
specifiškumas = {TN \over {TN + FP}} \cdot 100
\end{equation}

kur \textit{TN} yra tikrų neigiamų, \textit{TP} tikrų teigiamų, \textit{FP} klaidingai teigiamų, \textit{FN} klaidingai
neigiamų klasifikavimų skaičius.

Taip pat skaičiuojama ir ROC kreivė, AUC metrika.

\begin{ktutable}{webtrees_rezultatas}{Straipsnyje pateikiami matavimų rezultatai}
    \begin{tabular}{| l | c | c | c | c | }
     \hline
     \diagbox{Kategorija}{Metrika} & Tikslumas & Jautrumas & Specifiškumas & AUC \\ \hline
     Nekenksmingos & 98.3\% & 96.4\% & 96.5\% & 0.985 \\ \hline
     Kenksmingos & 92.9\% & 96.4\% & 96.5\% & 0.985 \\ \hline
    \end{tabular}
\end{ktutable}

Matavimų rezultatai pateikiami \vref{tab:webtrees_rezultatas}.


Straipsnyje lyginami įvairūs metodai skirti svetainių klasifikavimui į kenksmingas ir nekenksmingas. Formuluojama problema yra svetainių klasifikavimas į dvi grupes, naudojant metodus \cite{comp}
\begin{enumerate}
 \item K-artimiausų kaimynų (K-Nearest neighbours);
 \item Atraminių vektorių klasifikatorius (Support Vector Machine);
 \item Paprastasis Bayes klasifikatorius (Naive Bayes);
 \item K-vidurkių (K-Means);
 \item Affinity propagation metodas.
\end{enumerate}

Staripsnyje naudojami ir metodai mokomi su mokytoju ir be. Jų rezultatas yra lyginamas siekiant aptikti geriausiai tinkantį parinktai problemai spręsti \cite{comp}.

K-Artimiausų kaimynų (K-Nearest neighbours) metodas yra paremtas trimis elementais \cite{Wu2008}:
\begin{enumerate}
    \item Mokymo duomenų rinkinys;
    \item Elementų atstumo matas;
    \item Analizuojamų kaimynų kiekis $k$.
\end{enumerate}
Elementų klasifikacija vyksta balsavimo principu - išrenkama $k$ elementų, kurie yra artimiausi pagal parinktą artumo metriką, dominuojanti klasė išrinktame rinkinyje priskiriama klasifikuojamam elementui.

Pagrindiniai šio metodo trūkumai yra $k$, kaimynų skaičiaus, parinkimas. Per didelis $k$ lems gretimų klasių įtraukimą į balsavimo procesą, per mažas gali lemti blogą rezultatą esant triukšmui pradiniuose, apmokymo, duomenyse. Tai sprendžiama pridedant balso svorį \cite{Wu2008} kuris yra atvirkščiai proporcingas elemento atstumui nuo balsuojančio elemento. Taip pat svarbus atstumo mato parinkimas. Dažnai naudojamas Euklido atstumas, tačiau jis nėra tinkamas kai yra daug matmenų, ar jų reikšmės yra itin skirtingos \cite{Wu2008}. Tada verta naudoti kitus matus arba normalizuoti atributų vertes.

Šio metodo privalumas yra tai,  kad skaičiavimai atliekami tingiai. Pirminio modelio formavimo metu nėra atliekami jokie skaičiavimai, jie vykdomi tik klasifikuojant naujus duomenis \cite{Wu2008}.

Atraminių vektorių klasifikatoriaus (Support Vector Machine) metodas yra vienas iš patikimiausių klasifikavimo metodų \cite{Wu2008}. Jis pasižymi geru rezultatu net turint mažą kiekį mokymo duomenų bei nenukenčia dėl didelio matmenų kiekio \cite{Wu2008}. Šio metodo esmė yra duomenų atvaizdavimas kitoje erdvėje kurioje bus galima aptikti hiperplokštumą kuri skiria įrašų klases.

Atraminių vektorių klasifikatorius aprašomas\cite{comp}:
\begin{equation}
    h(x) = b + \sum_{n=1}^{N}y_i \alpha_i K(x, x_i),
\end{equation}
kur $h(x)$ yra elemento $x$ atstumas nuo klases skiriančios hiperplokštumos, $b$ yra papildomas svorio koeficientas, $\alpha$ yra hiperplokštumos paraštės korekcijos koeficientas mokymo imčiai, $N$ yra matmenų skaičius, $K$ yra branduolio funkcija kuri transformuoja elementus į atraminių vektorių erdvę, $x$ yra klasifikuojamas įrašas.

Apmokant atraminių vektorių klasifikatorių yra svarbu parinkti tinkamą branduolio funkciją, nuo jos itin priklauso modelio rezultatų tikslumas. Mokymas vykdomas parenkant branduolio funkciją ir pskaičiuojant tokius koeficientus $\alpha_i$ su kuriais gaunamas didžiausias riba tarp hiperplokštumos ir teisingai klasifikuotų mokymo duomenų erdvėje.

Šio metodo trūkumas yra didelio kiekio skaičiavimo resursų reikalavimas modelio apmokymo metu. Tačiau po modelio formavimo jis gali būti panaudojamas nereikalaujant daug resursų \cite{Wu2008}.

Paprastas Bajeso klasifikatorius yra paremtas Bajeso teorema. Laikoma kad visi įrašų matmenys priklauso nepriklauso vienas nuo kito \cite{comp}. Šio metodo privalumas yra paprastas modelio formavimas, nėra hiperparametrų kuriuos reikėtų derinti. Taip pat jo sudėtingumas mokymosi metu laiko atžvilgiu yra tiesinis, mokymosi imties dydžiui. Tai leidžia jį pritaikyti dideliems duomenų rinkiniams \cite{Wu2008}. Taip pat rezultatas yra pakankamai geras, ypač įvertinant reikalingą pastangų kiekį norint apmokyti modelį \cite{Wu2008}.

  Modelis formuojamas naudojantis Bajeso teorema ir sąlyginėmis tikimybėmis. Jei klasių aibė yra $i = 0, 1$ tik dviejų elementų, $P(i|x)$ yra tikimybė kad įrašas su verčių vektoriumi $x = (x_1, ..., x_p)$ priklauso klasei $i$, tai bet kokia monotoniška $P(i|x)$ gali būti naudojama klasifikavimui \cite{Wu2008}. Ši funkcija gali būti išreiškiama kaip santykis $P(1|x)/P(0|x)$. Tai gali būti išreiškiama kaip \cite{wu2008}:
    \begin{equation}
        \frac{P(1|x)}{P(0|x)} = \frac{f(x|1)P(1)}{f(x|0)P(0)}
    \end{equation}
  Šią funkciją klasifikavimui galima naudoti radus $f(x|i)$ funkciją. Paprastojo Bajeso klasifikavimo metode laikoma kad visi $x$ vektorių matmenys yra nepriklausomi, tada $f(x|i)$ prastinama į \cite{Wu2008}
  \begin{equation}
    f(x|i) = \prod_{j=1}^{p} f(x_j|i)
  \end{equation}
  kur visos $f(x_j|i)$ funkcijos yra įvertinamos atskirai, taip supaprastinant problemą į daug  vienmačių uždavinių.

  Šis metodas leidžia panaudoti nedidelį duomenų kiekį modelio apmokymui, apmokymas vyksta greitai ir paprastai, nereikalauja komplikuotų iteracinių schemų modelio formavimui \cite{Wu2008}. Tačiau vienas iš reikalavimų labiausiai ribojančių šį metodą yra visų duomenų matmenų nepriklausomumas tarpusavyje. Tačiau tai galima spręsti naudojant pirminį duomenų apdorojimą, pašalinant stipriai koreliuotas reikšmes \cite{Wu2008}. Taip pat metodas gali modeliuoti tik tiesines ribas tarp klasių, be papildomų modifikacijų jo pritaikyti netiesiniams uždaviniams neina \cite{comp}.

 K-vidurkių algoritmas yra iteracinis metodas skirtas skaidyti duomenų rinkinį į naudotojo parinktą klasterių skaičių $k$ \cite{Wu2008}. Algoritmo metu siekiama minimizuoti visų imties elementų atstumą iki jiems priskirtų klasterių centrų. Algoritmo iteracija susideda iš dviejų žingsnių:

 \textbf{Duomenų priskyrimas.} Visi objektai yra priskiriami jiems artimiausiam centroidui. Esant vienodiems atstumais centroidas parenkamas atsitiktinai. Pirmą kartą vykdant šį žingsnį centroidai parenkami atsitiktinai. Šio žingsnio rezultatas yra duomenys, suskaidyti į $k$ grupių.

 \textbf{Centrų tikslinimas.} Perskaičiuojamos grupių vidutinės parametrų vertės, parenkamas naujas centroidas esantis arčiausiai tikrojo grupės centro.
 Šie algoritmo žingsniai yra kartojami kol centroidai nebesikeičia \cite{Wu2008}. Atstumui įvertinti naudojamas Euklido atstumas \cite{comp}:
   \begin{equation}
   ||x_n - \mu_k||^2 = \sqrt{\sum_{i=1}^{D}(x_{ni} - \mu_{ki})^2}
   \end{equation}
   kur $\mu_k$ yra klasterio centroidas, $D$ yra duomenų rinkinio dydis, $x$ yra vienas iš duomenų rinkinio elementų.

   K vidurkių metodas turi trūkumų. Galutiniai klasteriai priklauso nuo pirminių taškų pasirinkimo. Netinkamai parinkti taškai gali lemti neoptimalų sprendimą. Taip pat k-vidurkių metodas nesugeba išskirti klasterių kurie nėra vienas nuo kito aiškiai atskirtos sferos erdvėje.

   Dėl vidurkio funkcijos naudojimo centroidų parinkimui rezultatas gali būti stipriai iškreiptas kelių išskirčių. Tai gali būti sprendžiama naudojant papildomą duomenų valymą, pirma suskaidant į daugiau klasterių, siekiant kad išskirtims būtų priskiriami maži klasteriai, ir vėliau jas prijungiant prie didesnių klasterių \cite{Wu2008}.

Affinity propagation yra vienas iš klasterizavimo metodų. Tai yra neprižiūrimas metodas \cite{comp}. Šio metodo įvestis yra panašumų matrica, kurioje saugoma visų duomenų įrašų porų panašumai $s[i, j]$ kur $i, j =  (1,  2, ..., N)$. Naudojant šį metodą siekiama surasti kiekvieno elemento duomenų aibėje atstovą. Tą galima analizuoti kaip žinučių tarp duomenų įrašų perdavimą \cite{fastprop}. Kiekvienai elementų porai $i$ ir $j$ egzistuoja dvi žinutės -- atsakomybė (responsibility), $r[i, j]$ siunčiama iš taško $i$ į $j$, kuri parodo sukauptus įrodymus apie elemento $j$ galimybę atstovauti elementą $i$ klasteryje.  Antra žinutė yra tinkamumas (availability) $a[i, j]$, tai žinutė siunčiama iš elemento $j$ elementui $i$ kuri parodo $j$ elemento tinkamumą atstovauti elementą $i$. Iš pradžių visos $a$ ir $r$ reikšmės nustatomos į nulį ir yra iteratyviai atnaujinamos pagal \cite{fastprop}:
 \begin{equation}
    r[i, j] = (1 - \lambda)\rho[i, j] + \lambda r[i, j]
    a[i, j] = (1 - \lambda)\alpha[i, j] + \lambda a[i, j]
 \end{equation}
 kur $\lambda$ yra konstanta naudojama sumažinti svyravimus, galimos reikšmės yra $ 0 \geq \lambda < 1 $ , o $\rho[i, j]$ ir  $\alpha[i, j]$ yra propaguojama atsakomybė ir propaguojamas tinkamumas \cite{fastprop}. Šie dydžiai gaunami pagal \cite{fastprop}:
 \begin{equation}
   \rho[i, j]=\left\{
                  \begin{array}{l r}
                    s[i,j] - max_{k \neq j} \{ {a[i, k] + s[i, k]} \} & (i \neq j) \\
                    s[i,j] - max_{k \neq j} \{ {s[i, k]} \} & (i = j) \\
                  \end{array}
                \right.
 \end{equation}
   ir \cite{fastprop}:
\begin{equation}
    \alpha[i, j] =  \left\{
         \begin{array}{l r}
            min \{ 0, r[i, j] + \sum_{k \neq i,j} max \{ 0, r[k, j] \} \}         & (i \neq j) \\
            \sum_{k \neq i,j} max \{ 0, r[k, j] \}  & (i = j) \\
         \end{array}
       \right.
\end{equation}
žinutės yra gaunamos iš atitinkamų propaguojamų žinučių. Galiausiai atstovas elementui $i$ yra apibrėžiamas kaip \cite{fastprop}
\begin{equation}
    argmax \{r[i, j] + a[i, j] : j = 1, 2 ..., N \}
\end{equation}

Šis metodas leidžia kontroliuoti suformuojamų klasterių kiekį per $\lambda$ vertę \cite{comp}. Šis metodas pasižymi geresne greitaveika lyginant su K-vidurkių metodu \cite{fastprop}.

Pritaikant šiuos metodus svetainės turi būti supaprastinamos iki tam tikrų savybių vektoriaus. Tyrimo metu nauodtas svetainių turinys, jų URL adresai \cite{comp}. Svetainių turinys, kuris yra anglų kalba buvo analizuojamas siekiant išgauti jų semantinę prasmę. Tam naudota Term Frequency - inverse document frequency (TFIDF) metodika. Taip pat į svetainių savybes įtraukta ir jų HTML kodo struktūra, jų adreso semantinė reikšmė, nuorodos į kitus tinklalapius, atvaizduoto tinklalapio vaizdas.

Duomenų rinkinys surinktas iš dviejų pagrindinių šaltinių, Alexa katalogo, iš kurio išgautos nekenksmingos svetainės, bei iš Phishtank žalingų svetainių registro. Surinkta $100000$ svetainių informacija.

\ktufigure{images/validation.png}{11 cm}{Modelio apmokymo ir validacijos proceso schema}

Modeliai apmokyti naudojant 70\% pradinių, sužymėtų duomenų apmokymui bei 30\% testavimui \ktufigref{images/validation.png}. Duomenų rinkinys atsitiktinai skaidomas į dvi dalis, mokymo ir validacijos. Mokymo duomenys naudojami modelio apmokymui, o validacijos duomenys naudojami įvertinti modelį. Taip yra išvengiama modelio persimokymo, galima teisingai suderinti modelio hiperparametrus. Kenksmingų ir nekenksmingų svetainių santykis išlaikytas vienodas testavimo ir mokymo imtyse.

\begin{ktutable}{webtrees_rezultatas}{Strapsnyje pateikiami modelių rezultatai}
    \begin{tabular}{l c c c c }
     \hline
       \diagbox{Metrika}{Modelis} & KNN & LS & RS & NB \\ \hline
        Tikslumas, naudojant 50 įrašų & 74\% & 80\% & 79\% & 77\% \\ \hline
        Tikslumas, naudojant 100 įrašų & 75\% & 82\% & 83\% & 78\% \\ \hline
        Tikslumas, naudojant 500 įrašų & 79\% & 86\% & 92\% & 78\% \\ \hline
        Tikslumas, naudojant 5000 įrašų & 91\% & 93\% & 97\% & 84\% \\ \hline
        Tikslumas, naudojant 100,000 įrašų & 95\% & 93\% & 98\% & 89\% \\ \hline
        AUC                               & 0.66  &  0.93 & 0.91 & 0.88 \\ \hline
        kur modeliai: \\
        KNN - K-artimiausių kaimynų \\
        LS - Tiesinio branduolio atraminių vektorių klasifikatorius \\
        RS - Radial Basis Function branduolio atraminių vektorių klasifikatorius \\
        NB - Paprastasis Bayes klasifikatorius
    \end{tabular}
\end{ktutable}

Modelių rezultatai pateikiami \vref{tab:webtrees_rezultatas} lentelėje. Modelis pasiekęs geriausius rezultatus yra tiesinio branduolio atraminių vektorių klasifikatorius. Deja šiame straipsnyje nėra vertinami sprendimų medžiai, dėl to nėra aišku, koks būtų jų rezultatas naudojant šiuos duomenų ir savybių rinkinius.
\ktusection{Analizės išvados}

Analizuojant egzistuojančią literatūrą aptikta, kad geri rezultatai klasifikuojant kenksmingas ir nekenksmingas svetaines yra pasiekiami naudojant įvairius metodus: C4.5 sprendimų medį \cite{trees}, atraminių vektorių mašinų \cite{comp}. Šie metodai bus mėginami bandant išrinkti geriausiai tinkantį parinktam panaudos atvejui.

Duomenų rinkiniai yra formuojami jungiant įvairių juodųjų sąrašų duomenis bei naudojant privačius duomenų rinkinius kurie taip pat atitinka juoduosius sąrašus \cite{trees, comp, webcop}.

Tinklalapių parametrai yra formuojami iš jų adresų (URL) \cite{trees, comp}. Interneto grafo struktūra naudojama straipsnyje \cite{linchpins} stengiantis identifikuoti klasterių centrus. Siūloma bandyti šią metriką pritaikyti ir klasifikavimo uždavinyje. Taip pat naudojami papildomi duomenys apie talpinančius serverius, įvairius svetainių metaduomenis, tačiau šie duomenys nėra viešai pasiekiami, ir jų surinkimui naudojami itin dideli skaičiavimo resursai -- \cite{linchpins} straipsnio atveju naudota 20 virtualių mašinų, renkančių duomenis 7 mėnesius. Dėl to naudojami tik URL bei svetainių grafo PageRank įverčiai.

Modeliai formuojami naudojant įvairius metodus, tačiau dalinimas į mokymo, testavimo ir validavimo imtis yra paprastas ir tinkamas, turint pakankamai didelį duomenų rinkinį. Kryžminė validacija gali būti naudojant esant mažesniems rinkiniams \cite{trees}. Modeliai vertinami naudojant įvairias metrikas: tikslumas, validumo vertinimas naudojant AUC, tikslumą, specifiškumą, jautrumą.

\newpage
\ktusection{Darbo tikslas ir uždavinys}
Pavojingi puslapiai skirstomi į skirtingas kategorijas pagal jų vykdomą \cite{tax} ir pagal jų paskirtį žalingų puslapių tinkle \cite{linchpins}. Jų aptikimui naudojami įvairūs metodai \cite{comp}. Šiame darbe bus tobulinamas tokių puslapių aptikimas, naudojant interneto grafo bei puslapių URL informaciją.

Dabro tikslas -- pasiūlyti metodiką ir sukurti įrankį kenksmingų ir nekenksmingų tinklalapių klasifikavimui, kur kenksmingi tinklalapiai yra pavojingų tinklalapių sąrašuose aptinkami adresai, ir įvertinti PageRank įverčio panaudojimą kenksmingų tinklalapių klasifikavimui.

Sprendžiami uždaviniai:
\begin{enumerate}
    \item apžvelgti klasifikavimo metodus naudojamus kenksmingų tinklalapių klasifikavimui;
    \item sukurti programinę įrangą skirtą tinklalapių klasifikavimui į kenksmingus ir nekenksmingus;
    \item įvertinti PageRank įverčio panaudojimą;
    \item pritaikyti metodą suformuotam duomenų rinkiniui.
\end{enumerate}


\newpage

\ktuchapter{Medžiagos ir tyrimų metodai}
Kuriama metodika tinklalapių klasifikavimui. Siekiamas galutinis rezultatas yra įrankis, kurio įvestis yra tinklalapio adresas, o rezultatas -- klasifikacija su jai priskirtu tikslumu. Tinklapio adresas (URL) yra pakankamas identifikuoti tinklalapius. Taip pat naudojantis juo galima pasiekti patį tinklapį, taip įrankiui paliekama galimybė pridėti papildomas savybes, tokias kaip tinklalapio turinio peržiūra.

Klasifikavimo modeliai savo bendroje formoje kaip įvestį naudoja parametrų vektorių, dažniausiai tai yra $n$ ilgio vektorius $X = (x_1, x_2, ... , x_n)$ žymintis klasifikuojamo objekto savybes. Savybės išreiškiamos skaitinėmis vertėmis. Klasifikatoriaus tikslas - kiekvienam vektoriui $X$ priskirti klasę $Y$. Dėl to vienas pirmų užduočių kuriuos reikia atlikti yra nagrinėjamo objekto, tinklalapio adreso, transformavimas į savybių vektorių $X$.

\ktusection{Savybių vektoriaus $X$ formavimas}

Tinklalapio savybių vektorius $X$ gaunamas iš tinklalapio adreso bei svetainės PageRank įverčių. Tinklalapio adresas yra simbolių eilutė, naiviu atveju būtų galima jį traktuoti kaip skaičių masyvą, tačiau taip būtų prarandama daug informacijos. Visą tinklapio adresą atitinkantis vektorius $X$ yra sudarytas iš savybių:
\begin{enumerate}
\item srities varde esantys žodžiai;
\item užklausoje esantys žodžiai;
\item taško simbolių skaičius adrese;
\item specialiųjų simbolių skaičius adrese (Naudojami simboliai:
\texttt{-},
\texttt{.},
\texttt{\_},
\texttt{\~},
\texttt{:},
\texttt{/},
\texttt{?},
\texttt{\#},
\texttt{[},
\texttt{]},
\texttt{@},
\texttt{!},
\texttt{\$},
\texttt{\&},
\texttt{'},
\texttt{(},
\texttt{)},
\texttt{*},
\texttt{+},
\texttt{,},
\texttt{;},
\texttt{=}
ir tarpo simbolis);
\item kenksmingų svetainių PageRank įvertis;
\item nekenksmingų svetainių PageRank įvertis.
\end{enumerate}.
Dėl žodžių naudojimo savybių vektoriuje jo ilgis priklauso nuo naudojamo duomenų rinkinio. Kiekvienam žodžiui rinkinyje naudojama po vieną lauką vektoriuje, dėl to jis gali tapti ganėtinai ilgas.

\ktufigure{images/url_struktura.png}{15 cm}{Tinklalapio adreso (URL) struktūra}

Pavyzdinis tinklalapio adresas yra pateikiamas \ktufigref{images/url_struktura.png} paveiksle. Čia matoma ir jo struktūra. Adreso pradžioje yra protokolas, dažniausiai aptinkami \texttt{http://} ar \texttt{https://}. Antras segmentas yra sritis. Kreipiantis į tinklalapį naudojamas sričių vardų serveris, kuris susieja simbolinį adresą, sritį, su skaitiniu adresu, pagal kurį galima kreiptis į svetainę talpinantį serverį. Sritis parenkama taip, kad atspindėtų svetainės turinį, būtų lengvai įsimenama naudotojams, bei būtų lengvai randama paieškos varikliuose. Sričių vardus suteikia sričių registrai, už tam tikrą mokestį. Kaina nustatoma pagal raktažodžių populiarumą. Paskutinis adreso segmentas yra užklausa. Ji leidžia svetainė talpinančiam serveriui pateikti tą dokumentą kuris atitinka tinklalapį kurį bando pasiekti naudotojas.

Išskiriamos sąvybės iš tinklalapio adreso yra:
\begin{enumerate}
\item srities varde esantys žodžiai;
\item užklausoje esantys žodžiai;
\item taško simbolių skaičius adrese;
\item specialiųjų simbolių skaičius adrese.
\end{enumerate}
Žodžiai adrese yra išskiriami kaip dviejų reikšmių sąvybės, $0$ jei žodis neegzistuoja adrese, $1$ jei žodis yra adrese. Laikoma, kad žodžius skiria specialieji simboliai.

Pagerank įverčiai gaunami iš orientuoto svetainių grafo $G = (V, E)$, kur $V$ yra šio grafo viršūnės o $E$ lankai. Grafo formavimas iš interneto svetainių yra pavaizduotas \ktufigref{images/hostgraph.png} paveiksle.
\ktufigure{images/hostgraph.png}{15 cm}{Svetainių grafo formavimas}

Grafo viršūnės $V$ yra svetainės, o lankai atitinka ryšius tarp svetainių. Tai yra nuorodos tarp svetainių aptinkamos svetainių tinklalapiuose. Šie lankai turi kryptį. Jei svetainės $v_a$ tinklalapyje yra nuoroda į svetainės $v_b$ tinklalapį laikoma, kad egzistuoja lankas $e_{ab}$ iš viršūnės $v_a$ į viršūnę $v_b$. Visų lankų svoris yra vienodas, tai yra jei keli svetainės $v_a$ tinklalapiai turi nuorodas į svetainės $v_b$ tinklalapius, vis vien laikoma kad yra tik vienas lankas $e_{ab}$. Galimi abipusiai svetainių ryšiai, tai yra jei egzistuoja lankas $e_{ab}$ lankas $e_{ba}$, kuris rodo ryšį iš viršūnės $v_b$ į viršūnę $v_a$ taip pat gali egzistuoti.

Svetainių grafas naudojamas skaičiuoti Pagerank įverčiams. Pagerank įvertis apibrėžiamas kaip \cite{pagerank}:
\begin{equation} \label{eq:pagerank}
    R(v) = (1 - d) + d \sum_{u \in B_{v}} \frac{R(u)}{N_u}
\end{equation}
kur $R(v)$ yra viršūnės $v$ PageRank įvertis, $d$ - slopinimo faktorius, $B_v$ viršūnės $u$ kurios turi lankus $e_{uv}$ su viršūne $v$. $N_u$ yra viršūnės $u$ išėjimo laipsnis, tai yra skaičius lankų kurie veda iš viršūnės $u$. Slopinimo faktorius $d$ gali turėti reikšmes tarp $0$ ir $1$.

PageRank įvertis negali būti tiesiogiai apskaičiuojamas pagal \ref{eq:pagerank} formulę. Naudojamas iteracinis metodas \ktualgoref{Pagerank} skaičiavimui:

\begin{ktualgo}{Pagerank}{Iteracinis PageRank įverčio skaičiavimo algoritmas}
\State{$R_0 \gets S$}
\Repeat
\For{$v_i$ grafe $G$}
\State{$r \gets 0$}
\For{$u \in B_{v_i}$}
\State{$r \gets r + \frac{R_i(u)}{N_u} $}
\EndFor{}
\State{$R_{i+1}(v_i) \gets (1 - d) + d r $}
\State{$\delta \gets \norm{R_i - R_{i+1}} $}
\EndFor{}
\Until{$\delta < e$}
\end{ktualgo}

Šis būdas apskaičiuoti PageRank priklauso nuo nuo kelių parametrų \vref{tab:pagerank_params}. Tai yra $S$, nulinės iteracijos įvertis. Šios iteracijos metu nustatytos reikšmės bus naudojamos tolimesnių viršūnių įverčiui rasti. Nustatant šį vektorių galima koreguoti kurios viršūnės yra svarbios skaičiavimų pradžioje. Šio vektoriaus keitimas leidžia pritaikyti PageRank įverčio rezultatą. Parametras $d$ yra slopinimas, nuo jo priklauso metodo konvergavimo greitis. Viskas vykdoma kol pasiekiamas pasirinktas konvergavimo lygmuo, kurį galima aptikti pagal tai, jog iteruojant įvertis kinta nežymiai. Minimali iteracijos įverčio pakitimo riba nustatoma kintamuoju $e$.

\begin{ktutable}{pagerank_params}{Iteracinio PageRank įverčio skaičiavimo algoritmo parametrai}
    \begin{tabular}{ | r | l | }
        \hline
        Parametras & Reikšmė \\ \hline
        $G$ & Grafas kurio įverčiai skaičiuojami \\ \hline
        $S$ & Pirminės įverčio reikšmės \\ \hline
        $d$ & Slopinimo faktorius \\ \hline
        $e$ & Siekiama klaida tarp iteracijų \\ \hline
    \end{tabular}
\end{ktutable}

PageRank įvertis skaičiuojamas du kartus, naudojant skirtingas pradines reikšmes. Skaičiuojant nekenksmingų svetainių įvertį mokymo imties svetainių viršūnėms priskiriama nenulinė reikšmė, o visoms kitoms svetainėms nustatoma nulinė pradinė reikšmė. Skaičiuojant kenksmingų svetainių įvertį nenulinės reikšmės priskiriamos kenksmingų tinklalapių svetainių viršūnėms. Priskiriamos vertės yra vienas iš modelio mokymo metu derinamų hiperparametrų.

Gavus PageRank įverčius, jie yra naudojami suformuoti vektoriui $X$, taip baigiant tinklalapių nuorodų transformavimą į klasifikavimo modeliams pritaikytą vektorių.

\ktusection{Klasifikavimo modeliai}

Lyginami trys skirtingi klasifikavimo modeliai:
\begin{enumerate}
\item C4.5 sprendimų medis
\item atraminių vektorių mašinos
\item dirbtinis neuronų tinklas
\end{enumerate}

\ktusubsection{C4.5 sprendimų medžio klasifikatorius}
C4.5 sprendimų medis yra sprendimų medžio tipo klasifikatorius kurio formavimui naudojamas C4.5 algoritmas.
Sprendimų medis yra programa, suformuota iš tam tikrų testų, kurie tikrina nagrinėjamo objekto parametrus,
ir pagal juos parenkama tam tikra klasė.

\ktufigure{images/sprendimu_medis.png}{8 cm}{Pavyzdinis sprendimų medis}

Pavyzdinis sprendimų medis klasifikuojantis vektorių $X$ į klases $Y = (y_1, y_2, y_3)$ pateikiamas
 \ktufigref{images/sprendimu_medis.png} paveiksle. Jį sudaro testų rinkinys $T$.

Sprendimų medį sudarantys testai yra formuojami modelio apmokymo metu. Mokymui naudojamas duomenų
rinkinys $H$ kurį sudaro mokymosi duomenys su jiems priskirtomis klasėmis. Medžio formavimo metu
parenkami testai kurie maksimizuoja funkciją:
\begin{equation} \label{eq:gainratio}
    santykinisIšlošis(t_i) = \frac{išlošis(t_i)}{santykinėEntropija(t_i)}
\end{equation}
Santykinis išlošis priklauso nuo informacijos išlošio, kuris būtų maksimizuojamas naudojant ID3
algoritmą bei santykinės entropijos. Santykinė entropija naudojama siekiant sumažinti informacijos
išlošio iškreipimą kai testas skaido rinkinį į daug skirtingų klasių \cite{c45}
\begin{equation}
    išlošis(t_i) = entropija(H) - entropija_{t_i}(H)
\end{equation}
informacijos išlošis vertina duomenų rinkinio klasių atsiskyrimą - skaičiuojama kiek informacijos
suteikia padalinimas naudojant nagrinėjamą testą

\begin{equation}
    entropija_{t_i} = \sum_{j=1}^{n} \frac{|H_i|}{H} \times entropija(H_i)
\end{equation}

\begin{equation}
    entropija(S) = - \sum_{j=1}^{k} \frac{freq(C_j, S)}{|S|} \times log_2 \big( \frac{freq(C_j, S)}{|S|} \big)
\end{equation}

\begin{equation}
    santykinėEntropija(S) =  \sum_{j=1}^{n} \frac{|H_i|}{H} \times log_2 \big( \frac{|H_i|}{H} \big)
\end{equation}

kur $freq(C_j, S)$ žymi klasės $C_j$ pasikartojimo kiekį rinkinyje $S$.
Aptikus testą kuris maksimizuoja santykinį išlošį \ref{eq:gainratio} testas yra įtraukiamas į medį.
Mokymo duomenų rinkinys $H$ yra skaidomas pagal testą į kelis rinkinius\cite{c45}. Šie duomenų
rinkiniai priskiriami medžio šakoms atitinkančioms testo rezultatus ir rekursiškai tęsiamas medžio formavimas.

C4.5 medžio formavimo algoritmas palaiko diskrečių reikšmių parametrus bei tolygius kintamuosius.
Testai vykdomi diskretiems parametrams sukuria po vieną šaką kiekvienai galimai parametro reikšmei.

Testai tolydiems parametrams vykdomi parenkant ribinę vertę $Z$, su kuria bus lyginama parametro
$A$ vertė \cite{c45}. Kadangi apmokymui naudojamas baigtinis duomenų rinkinys, jame yra baigtinis
kiekis galimų tolydaus parametro $A$ reikšmių $\{v_1, v_2, ..., v_m\}$. Ribinei vertei $Z$ priskiriama
viena iš egzistuojančių parametro $A$ verčių  $v_i$. Parenkama vertė kuri maksimizuoja
santykinį išlošį \ref{eq:gainratio}.

Suformavus medį vykdomas medžio mažinimas. Dalis šakų yra sukeliančios persimokymą - tai yra
jos yra per daug pritaikytos mokymo imčiai \cite{c45}. Šios šakos yra aptinkamos skaičiuojant
tikėtiną klaidų dažnį medžiui be tam tikro testo ir su potencialiai pertekliniu testu.

Šis metodas neturi hiperparametrų kuriuos būtų galima derinti.
\ktusubsection{Atraminių vektorių mašinų klasifikatorius}
\ktusubsection{Dirbtinio neuroninio tinklo klasifikatorius}
\ktusection{Modelio mokymas}

Modelio apmokymas vykdomas programinės įrangos paketu Vowpal Wabbit \cite{vw}.

\ktusubsection{Sprendimų medžio modelio apmokymas}

%60k imtis, linear, --l2 1e-08 -b 31, runtime 6m
%
%Lambda = 0
%Kernel = linear
%Num support = 8221
%Number of kernel evaluations = 0 Number of cache queries = 0
%Total loss = 0
%Done freeing model
%Done freeing kernel params
%Done with finish
%ACC    0.86527   pred_thresh  0.500000
%PPV    0.00000   pred_thresh  0.500000
%NPV    0.86527   pred_thresh  0.500000
%SEN    0.00000   pred_thresh  0.500000
%SPC    1.00000   pred_thresh  0.500000
%PRE    0.00000   pred_thresh  0.500000
%REC    0.00000   pred_thresh  0.500000
%PRF    0.00000   pred_thresh  0.500000
%LFT    0.00000   pred_thresh  0.500000
%SAR    0.66607   pred_thresh  0.500000 wacc  1.000000 wroc  1.000000 wrms  1.000000
%
%ACC    0.13473   freq_thresh  0.000000
%PPV    0.13473   freq_thresh  0.000000
%NPV    0.00000   freq_thresh  0.000000
%SEN    1.00000   freq_thresh  0.000000
%SPC    0.00000   freq_thresh  0.000000
%PRE    0.13473   freq_thresh  0.000000
%REC    1.00000   freq_thresh  0.000000
%PRF    0.23746   freq_thresh  0.000000
%LFT    1.00000   freq_thresh  0.000000
%SAR    0.42256   freq_thresh  0.000000 wacc  1.000000 wroc  1.000000 wrms  1.000000
%
%ACC    0.13473   max_acc_thresh  0.000000
%PPV    0.13473   max_acc_thresh  0.000000
%NPV    0.00000   max_acc_thresh  0.000000
%SEN    1.00000   max_acc_thresh  0.000000
%SPC    0.00000   max_acc_thresh  0.000000
%PRE    0.13473   max_acc_thresh  0.000000
%REC    1.00000   max_acc_thresh  0.000000
%PRF    0.23746   max_acc_thresh  0.000000
%LFT    1.00000   max_acc_thresh  0.000000
%SAR    0.42256   max_acc_thresh  0.000000 wacc  1.000000 wroc  1.000000 wrms  1.000000
%
%PRB    0.13473
%Warning: There are more than 50000 ties in your prediction, using pessimistic ordering in Average Precision.
%APR    0.07062
%ROC    0.50000
%R50    0.00031
%RKL    90724
%TOP1   0.00000
%TOP10  0.00000
%SLQ    0.53370 Bin_Width  0.010000
%CXE   8999999999997488077981482897577656465333014215252573954519616480358124176326309577600924132162142208.00000
%RMS    1.00000
%CA1    0.05263 19_0.05_bins
%CA2    1.00000 Bin_Size 100
%371 - START_TIME
%
\newpage

\ktuchapter{Metodo taikymas ir rezultatai}

Aprašyto metodo pritaikytas realiam uždaviniui sudarytas iš dviejų žingsnių. Tai yra modelio apmokymas bei apmokyto modelio naudojimas. Modeliui mokyti reikalingas duomenų rinkinys.

\ktusection{Duomenų rinkinys}

Reikalingas duomenų rinkinys yra sudarytas iš trijų komponentų:
\begin{enumerate}
    \item Svetainių grafo duomenys;
    \item kenksmingų tinklalapių sąrašas;
    \item nekenksmingų tinklalapių sąrašas.
\end{enumerate}

\ktusubsection{Svetainių grafas}
Interneto grafo duomenys naudojami iš \cite{webgraph} projekto teikiamų duomenų rinkinių. Naudojamas grafą sudaro 22 milijonai viršūnių bei 123 milijonų briaunų. Grafas yra pateikiamas WebGraph formatu. Šis formatas yra optimizuotas grafo užimamos vietos diske atžvilgiu. Jame yra saugomi grafo viršūnių numeriai bei briaunos. Viršūnes atitinkančios svetainės yra randamos naudojant pagalbinį failą talpinantį svetainės adresą bei jos eilės numerį grafe. Grafas gali būti nuskaitomas naudojant atviro kodo biblioteką sukurtą pagal \cite{boldi2004webgraph} straipsnį.

\ktusubsection{Kenksmingų tinklalapių sąrašas}
Kenksmingi tinklapiai surinkti naudojant egzistuojančius juoduosius sąrašus \cite{mal1}, \cite{mal2} \cite{mal3} \cite{mal4}. Taip suformuotas 48761 kenksmingų tinklalapių sąrašas. Jų tipas, pagal vykdomą ataką, nėra išskiriamas. Šių puslapių duomenų rinkiniai yra naudojami kuriant juoduosius sąrašus. Visi rinkiniai sujungiami į vieną tekstinį failą, kuriame vienoje eilutėje suagoma nuoroda į vieną kenksmingą tinklalapį.

\ktusubsection{Nekenksmingų tinklalapių sąrašas}

Nekenksmingų tinklapių sąrašas yra formuojamas iš \cite{webgraph} projekto teikiamų duomenų. Naudojamas tinklalapių grafų duomenų rinkinio tinklalapių sąrašas. Šiame duomenų rinkinyje yra apie 1.7 milijardo tinklalapių. Suspausti šie duomenys diske sudaro apie 20GB informacijos. Išskleidus tai sudarytų apie 133GB duomenų. Tai yra gana daug norint juos apdoroti. Viso sąrašo naudojimas sukeltų per didelį klasių disbalansą, dėl to parenkama dalis įrašų. Įrašai yra išrenkami naudojant scenarijų:
\ktusrcr{whitelist.sh}{sources/whitelist.sh}{bash}
Šiuo atveju naudojamos atviro kodo programos zcat, parallel \cite{parallel}, awk. Šios programos parinktos dėl galimybės failų turinį apdoroti kaip duomenų srautą.   Pirmas programos fragmentas skaito visų failų turinį, antras fragmentas skaldo failų turinį į fragmentus kurie apdorojami lygiagrečiai. Kiekvienas lygiagretus procesas išrenka eilutes. Eilutės perkėlimo į galutinį failą tikimybė $P = 0.0005798$. Šis tikimybė siekiant suformuoti duomenų rinkinį kurio 5\% sudaro kenksmingos svetainės, 95\% nekenksmingos svetainės. Lygiagretus vykdymas naudojamas dėl atsitiktinių skaičių generavimo vykdymo metu. Naudojant vieną procesą kompiuterio resursai būtų naudojami neefektyviai, šiuo atveju yra išnaudojami visi turimi skaičiavimo resursai - nauji procesai yra kuriami tol kol pilnai naudojami visi procesoriaus branduoliai. Galiausiai gaunamas nekenksmingų svetainių sąrašas. Parinktų tinklalapių pasiskirstymas pradiniame duomenų rinkinyje yra tolydus.

Dalis tinklalapių naudoja papildomą parametrų kodavimą URL-kodavimo principu. Šis kodavimas tinklalapių adresuose neleistinus simbolius ar rezervuotus simbolius keičia specialiais procento simboliu prasidedančiais kodais. Adresai yra dekoduojami naudojant prieduose pateiktas programas.
\ktusection{Modelio mokymas}

Modelio apmokymas vykdomas programinės įrangos paketu Vowpal Wabbit \cite{vw}.

\ktusubsection{Sprendimų medžio modelio apmokymas}

%60k imtis, linear, --l2 1e-08 -b 31, runtime 6m
%
%Lambda = 0
%Kernel = linear
%Num support = 8221
%Number of kernel evaluations = 0 Number of cache queries = 0
%Total loss = 0
%Done freeing model
%Done freeing kernel params
%Done with finish
%ACC    0.86527   pred_thresh  0.500000
%PPV    0.00000   pred_thresh  0.500000
%NPV    0.86527   pred_thresh  0.500000
%SEN    0.00000   pred_thresh  0.500000
%SPC    1.00000   pred_thresh  0.500000
%PRE    0.00000   pred_thresh  0.500000
%REC    0.00000   pred_thresh  0.500000
%PRF    0.00000   pred_thresh  0.500000
%LFT    0.00000   pred_thresh  0.500000
%SAR    0.66607   pred_thresh  0.500000 wacc  1.000000 wroc  1.000000 wrms  1.000000
%
%ACC    0.13473   freq_thresh  0.000000
%PPV    0.13473   freq_thresh  0.000000
%NPV    0.00000   freq_thresh  0.000000
%SEN    1.00000   freq_thresh  0.000000
%SPC    0.00000   freq_thresh  0.000000
%PRE    0.13473   freq_thresh  0.000000
%REC    1.00000   freq_thresh  0.000000
%PRF    0.23746   freq_thresh  0.000000
%LFT    1.00000   freq_thresh  0.000000
%SAR    0.42256   freq_thresh  0.000000 wacc  1.000000 wroc  1.000000 wrms  1.000000
%
%ACC    0.13473   max_acc_thresh  0.000000
%PPV    0.13473   max_acc_thresh  0.000000
%NPV    0.00000   max_acc_thresh  0.000000
%SEN    1.00000   max_acc_thresh  0.000000
%SPC    0.00000   max_acc_thresh  0.000000
%PRE    0.13473   max_acc_thresh  0.000000
%REC    1.00000   max_acc_thresh  0.000000
%PRF    0.23746   max_acc_thresh  0.000000
%LFT    1.00000   max_acc_thresh  0.000000
%SAR    0.42256   max_acc_thresh  0.000000 wacc  1.000000 wroc  1.000000 wrms  1.000000
%
%PRB    0.13473
%Warning: There are more than 50000 ties in your prediction, using pessimistic ordering in Average Precision.
%APR    0.07062
%ROC    0.50000
%R50    0.00031
%RKL    90724
%TOP1   0.00000
%TOP10  0.00000
%SLQ    0.53370 Bin_Width  0.010000
%CXE   8999999999997488077981482897577656465333014215252573954519616480358124176326309577600924132162142208.00000
%RMS    1.00000
%CA1    0.05263 19_0.05_bins
%CA2    1.00000 Bin_Size 100
%371 - START_TIME
%
\ktusection{Modelio naudojimas kenksmingų interneto tinklalapių identifikavimo programos kūrimui}

Apmokytas modelis yra paruoštas naudojimui, tačiau jis nėra itin patogus. Naudojamos programinės įrangos modelis gali
būti panaudojamas per komandinės eilutės sąsają, pateikiant duomenis mokymo metu naudotu formatu. Modelio grąžinamas
rezultatas taip pat nėra lengvai suvokiamas. Dėl to yra kuriama taikomoji programa kuri supaprastina modelio
panaudojimą realiose, praktinėse situacijose.

Kuriama sistema vadinama toliau vadinama \textbf{kenksmingų interneto tinklalapių identifikavimo programa}.

\begin{ktutable}{sasaja}{Kenksmingų interneto tinklalapių identifikavimo programos sąsaja}
    \begin{tabular}{|l|l|p{5cm}|p{5cm}|}
    \hline
        Kelias & Metodas & Parametrai & Rezultatas \\ \hline
        \texttt{/classify} & \texttt{GET} & \texttt{url} klasifikuojamo tinklalapio adresas & slankaus kablelio skaičius - klasifikavimo rezultatas \\ \hline
    \end{tabular}
\end{ktutable}

Kenksmingų interneto tinklalapių identifikavimo programa pritaikyta naudoti http protokolu, naudojama REST architektūra. Sąsaja pateikta \vref{tab:sasaja}
lentelėje. Tinklalpio kurį norima klasifikuoti adresas yra pateikiamas klasifikavimo programai, grąžinamas
rezultatas yra klasifikavimo modelio pateiktas įvertis.

Kenksmingų interneto tinklalapių identifikavimo programos architektūra pateikiama paveiksle \ktufigref{images/architektura.png}

\ktufigure{images/architektura.png}{12 cm}{Kenksmingų interneto tinklalapių identifikavimo programos architektūra}

Kenksmingų interneto tinklalapių identifikavimo programa susideda iš dviejų komponentų:
\begin{enumerate}
\item sąsaja,
\item branduolys.
\end{enumerate}
Sąsaja yra komponentas kurio funkcija yra pateikti branduolio suteikiamą funkcionalumą naudotojams. Šiuo atveju
tai yra http protokolu veikianti sąsaja. Šis komponentas atsakingas už užklausos perdavimą branduolio komponentui,
branduolio komponente esančio klasifikatoriaus rezultato pateikimui naudotojui tinkamu formatu. Programinio
įrankio išeities kodas pateikiamas prieduose \vref{priedas:frontend}.

Branduolio komponente vykdomas modelio pritaikymas. Jį sudaro klasė \texttt{ClassifierController} bei duomenų
saugyklos kenksmingų ir nekenksmingų svetainių PageRank įverčiams.

Kenksmingų interneto tinklalapių identifikavimo programoje atskirti komponentai siekiant supaprastinti programos modifikacijas.. Atskiriant branduolio elementą kuriame yra
vykdoma verslo logika, matematiniai skaičiavimai sistema yra paruošiama greitam plėtimui, lengvam migravimui
tarp palaikomų ir naudojamų vartotojo sąsajos protokolų. Šiuo atveju vartotojo sąsaja nepriklauso nuo
branduolio, tik nuo klasės \texttt{ClassifierController} implementacijos.

Lyginant su alternatyvia sistema \cite{gapi} yra matomi kenksmingų interneto tinklalapių identifikavimo programos privalumai:
 \begin{enumerate}
    \item galima naudoti alternatyvi sąsaja, pakeitus sąsajos komponento implementaciją;
    \item klasifikavimo modelis gali būti keičiamas;
    \item nereikalauja interneto prieigos.
 \end{enumerate}

Pirminė programos versija yra kurta siekiant įrodyti jos veikimo koncepciją. Dėl to kai kurios sistemos vietos
yra optimizuotos programavimo laiko minimizavimo atžvilgiu. Plėtojant programinę įrangą vertėtų programą
skaidyti į kelis atskirtus komponentus, pagal naudojamų resursų tipą. PageRank įverčiai turi būti iškeliami
į duomenų bazę, mašininio mokymo modelis iškeliamas į atskirą vykdomąją aplikaciją kuri gali būti plečiamas
esant reikalui ar skaičiavimo resursų trūkumui.
\ktusection{Praktinis kenksmingų interneto tinklalapių identifikavimo programos taikymas}

Sukurtas įrankis bandytas panaudoti naudojant kelias nekenksmingas interneto svetaines. Dalis iš bandytų
tinklalapių pateikiama lentelėje \vref{tab:bandymas}.

\begin{ktutable}{bandymas}{Kenksmingų interneto tinklalapių identifikavimo programos bandymo rezultatai}
    \begin{tabular}{|p{5 cm}|p{5 cm}|p{3 cm}|}
    \hline
    Tinklalapis & Tinklalapio adresas & Kenksmingas \\ \hline
    KTU titulinis tinklalapis & http://ktu.edu/ & Ne \\ \hline
    Wikipedia straipsnis apie dramblius & https://lt.wikipedia.org/ wiki/Drambliai & Ne \\ \hline
    Socialinis tinklas Facebook & https://facebook.com & Taip \\ \hline
    Google paieškos variklis & https://google.com & Taip \\ \hline
    15 min laikraštis & https://15min.lt & Taip \\ \hline
    \end{tabular}
\end{ktutable}

Toks kenksmingų interneto tinklalapių identifikavimo programos bandymas parodo trūkumų. Tam tikri tinklalapiai kaip Google, 15min kurie nėra kenksmingi,
yra klasifikuojami klaidingai. Tačiau validavimo imties klasifikavimo kokybės metrikos rodo gerus rezultatus - \textbf{tikslumas 0.976, specifiškumas 0.982, jautrumas 0.932}.

Programos trūkumas yra tai, kad iš logistinės regresijos modelio negalima paprastai gauti priežasčių, dėl ko tinklalapis buvo klasifikuotas kaip kenksmingas ar nekenksmingas. Tačiau kenksmingų interneto tinklalapių identifikaivmo programa turi ir privalumų. Kenksmingų tinklalapių klasifikavimas vykdomas pagal mokymo duomenų rinkinį, tai leidžia pakeitus kenksmingų tinklalapių sąrašą identifikuoti kitos klasės tinklalapius, pavyzdžiui pagal jų tematiką. Taip pat sistemos veikimas yra aiškus - komercinės sistemos nepateikia tikslaus klasifikavimo metodo. Šią programą galima lengvai modifikuoti, ar keičiant sistemos sąsają ar klasifikavimo modelį. Taikant programą praktikoje būtina atsižvelgti į šiuos trūkumus bei privalumus.

\ktuchapter{Išvados}

\begin{enumerate}[label=\arabic*.]
\item Apžvelgus kensmingų tinklalapių klasifikavimo metodus pasiūlyta naudoti PageRank įverčius topologinei svetainių grafo informacijai panaudoti klasifikavimo modelių formavime
\item Sudaryta kenksmingų tinklalapių klasifikavimo metodika
\item Įgyvendinta pasiūlyta metodologija sukuriant mašininio mokymo modelius bei sukurtas programinis įrankis praktiniam modelio panaudojimui
\item Įvertintas sukurto įrankio praktinis panaudojimas bei įrankio trūkumai: lankstumo bei aiškumo trūkumas klasifikuojant tam tikrus populiarius tinklalapius.
\end{enumerate}

\newpage

\begin{ktuliterature}
    \printbibliography{}
\end{ktuliterature}

\newpage

\begin{ktuappendices}
\ktusection{Modelio mokymo duomenų paruošimo programos išeities kodas}
\label{priedas:executor}
\ktusrcr{UrlBuilder.scala}{apps/src/main/scala/andriusdap/orbweaver/executor/UrlBuilder.scala}{scala}
\ktusrcr{Pagerank.scala}{apps/src/main/scala/andriusdap/orbweaver/executor/Pagerank.scala}{scala}
\ktusrcr{App.scala}{apps/src/main/scala/andriusdap/orbweaver/executor/App.scala}{scala}


\ktusection{Modelį naudojančio klasifikavimo programinio įrankio išeities kodas}
\label{priedas:frontend}
\ktusrcr{ClassifierController.scala}{apps/src/main/scala/andriusdap/orbweaver/controllers/ClassifierController.scala}{scala}
\ktusrcr{Classifier.scala}{apps/src/main/scala/andriusdap/orbweaver/core/Classifier.scala}{scala}
%\ktusrcr{routes}{apps/src/main/resources/routes}{scala}
%\ktusrcr{application.conf}{apps/src/main/resources/application.conf}{scala}
%\ktusrcr{application.properties}{apps/src/main/resources/application.properties}{scala}
\ktusrcr{test.sh}{apps/test.sh}{scala}

\end{ktuappendices}
\end{document}
