\ktuchapter{Literatūros apžvalga}

\ktusection{Pavojingos svetainės}

Didelė dalis interneto svetainių yra komerciniai projektai. Naudojami įvairūs pajamų šaltiniai.
Tai gali būti reklamos, vykdomi pardavimai, prenumeravimo principu veikiančios svetainės.
Šie būdai yra vartotojų matomi ir suprantami. Tačiau egzistuoja ir svetainės kurios pajamas
generuoja naudotojams kenkiančiais metodais. Vagiamos naudotojų tapatybės, banko sąskaitų, kreditinių kortelių
duomenys, užvaldomi kompiuteriai siekiant pasipelnyti. Tai yra pavojingos svetainės.

Pavojingos svetainės taip gali būti talpinamos ir ne savo noru.
Naudojantis spragomis programinėje įrangoje galima įterpti žalingą kodą į kitas svetaines,
taip pasinaudojant jų naudotojais ir jų pasitikėjimu svetaine.

Egzistuoja įvairūs būdai saugoti naudotojus nuo pavojingų svetainių \cite{trees}.
Vienas iš naudojamų variantų yra pavojingų svetainių blokavimas.
Naudotojui naršant internete svetainės analizuojamos pagal tam tikras savybes.
Svetainę galima analizuoti pagal įvairias metrikas \cite{tangled},
talpinančio serverio parametrus, raktažodžius adrese, asinchroniškai įkeliamus resursus.

Pavojingų svetainių aptikimo modulis yra pagrindinis tokios filtravimo sistemos komponentas.
Šis modulis sprendžia svetainių klasifikavimo problemą. Tam gali būti naudojami įvairūs mašininio mokymosi algoritmai.
Vienas iš puikiai tinkančių šiam panaudos atvejui \cite{trees} yra sprendimų medis.
Pagrindiniai jo privalumai yra didelis veikimo greitis, nesudėtingas derinimas - nėra daugybės hyperparametrų.

Modelio apmokymui naudojami egzistuojantys juodieji svetainių sąrašai.
Adresai sąrašuose yra skaldomi į žetonus kurie sudaryti iš žodžių, simbolių, skaičių.
Taip pat galima įtraukti ir svetainės metaduomenis - dydį, serverio duomenis, naudojamos programinės įrangos versijas.
Tačiau įprasti sprendimų medžiai nagrinėja vieną svetainę \cite{trees}.
Taip ignoruojama svarbiausia interneto savybė -- svetainės talpina nuorodas į kitas svetaines.
Šios nuorodos tiekia papildomą informaciją apie svetainę,
jos turinys bei rangas interneto grafe priklauso nuo svetainių kurios turi nuorodą į ją.
Šiuo principu yra pagrystas PageRank algoritmas \cite{pagerank} varantis "Google" paieškos variklį.
Laikant, kad svarbios svetainės turi nuorodas į kitas svarbias svetaines kyla mintis,
kad tai turi galioti ir pavojingoms svetainėms.
Šios svetainės siekia pritraukti žmones, dėl to turi būti optimizuotos SEO prasme.


\ktusection{Žalingų svetainių aptikimas skenuojant iš viršaus žemyn}

Ryšiais tarp žalingų svetainių galima naudotis aptinkant kaimynines žalignas svetaines \cite{webcop}.
To tyrimo metu siekiama rasti žalingą programinę įrangą platinančias svetaines. Iškskiriami du žalingų svetainių tipai --
tai yra nuorodos į žalingus failus bei svetainės talpinančios žalingos programinės įrangos nuorodas.
Priešingai nei daugelyje kitų metodikų naudojama ne iš viršaus žemyn metodika, skenuojant interneto grafą ir ieškant kenkėjų,
o iš apačios viršun metodas.

Ieškant žalingų svetainių iš viršaus žemyn yra einama per interneto grafą ir tikrinama ar rasti puslapiai yra žalingi \cite{webcop}.
Tai aptinkama puslapių lankymą vykdant iš mašinų kurių būsena yra gerai žinoma. Aplankius svetainę tikrinama kaip pakito
mašinos būsena po aplankymo. Tam naudojamos virtualios mašinos, kurios turi būti kuriamos iš naujo lankant naują puslapį.
Tai reikalauja itin daug kompiuterinių resursų -- virtualios mašinos perkūrimo procesas yra lėtas lyginant su
internetinio puslapio aplankymu. Taip pat po puslapio aplankymo virtuali mašina turi būti ištirta. Tyrimo metu tikrinama
ar nenutiko netikėti pakitimai operacinėje sistemoje, bei vartotojui prieinamuose failuose.

Skenavimas iš viršaus žemyn taip pat nesinaudoja interneto socialinio grafo savybėmis. Kenkėjiškos svetainės yra linkusios
turėti ryšius su kitomis kenkėjiškomis svetainėmis, tik nedidelė dalis vartotojams saugių svetainių turi nuorodas į
kenkėjiškas svetaines. Taip pat kenkėjiškos svetainės linkusios dažnai keisti savo domenus \cite{webcop} bei juos
generuoti automatiškai \cite{trees}. Tai reikalauja dažnai perskenuoti interneto grafą iš naujo analilzuojant visus
rastus puslapius.

Galima alternatyva yra iš apačios žemyn veikiantis metodas. Šiuo metodu vietoje vieno multigrafo naudojamos dvi duomenų
struktūros. Tai yra interneto grafas bei žalingų svetainių sąrašas. Interneto grafas formuojamas greičiau nei skenavimo
svetaines iš viršaus žemyn metodu, nes grafo formavimo metu nėra siekiama identifikuoti pavojingus puslapius. Antra naudojama
duomenų struktūra yra žalingų svetainių sąrašas. Žalingų svetainių sąrašas yra formuojamas naudojant atskirą nuo
interneto svetainių skenavimo procesą. Tam naudojami įvairūs Microsoft kompanijos saugumo įrankiai. Jų naudojimo metu
yra identifikuojamos žąlingos naudotojų parsiunčiamos programos. Žinant iš kokios nuoroduos buvo gautas kenkėjiškas failas
jis yra registruojamas centrinėje duomenų bazėje.

Interneto grafas yra jungiamas su žalingus failus platinančiom svetainėmis ir surandamos viršūnės turinčios lankus į žalingus failus.
Taip yra išskiriami du kenkėjiškų svetainių tipai:
\begin{enumerate}[label=\alph*]
    \item Nuorodos į žalingus failus
    \item Žalingų failų katalogas
\end{enumerate}
Naudojantis interneto svetainių grafu randamos gretimos svetinaės, tikėtina, kad šios svetainės yra kenksmingos \cite{webcop}.
Tokia metodika leidžia uždavinį skaidyti į dvi dalis - pavienių svetainių kenksmingumo tyrimas bei jų kaimynų aptikimas grafe.
Grafo struktūra naudojama tik gretimas svetaines grafe -- analizė nėra vykdoma toliau nei vienas žingsnis.

Svarbi \cite{webcop} tyrimo dalis yra klaidingai žalingai įvertintų svetainių įvertinimas. Vienas iš rodiklių
padedančių identifikuoti klaidas yra grafo viršūnės laipsnis. Nežalingų svetainių grafo viršūnės laipsnis yra
eksponentiškai didesnis nei kenkėjiškų svetainių.

Straipsnyje \cite{webcop} naudojamame metode yra keli trūkumai. Naudojamas daug resursų interneto grafo formavimui,
tačiau visa grafo suteikiama informacija nėra panaudojama. Dėl to kad angrinėjami tik artimiausi kaimynai, prarandama
daug informacijos apie kaimynus kurie yra nutolinami dirbtinai -- įterpiant papildomų nukreipimų kuriuos įprastos
naršyklės išspręndžia naudotojui nepastebint. Taip pat naudojamas sudėtingas būdas aptikti pradinį žalingų svetainių
sąrašą -- tam panaudojama Microsoft kuriamų antivirusinių programų naudotojų duomenys.
Tai negali būti replikuota be daugybės naudotojų.


\newpage