\ktuchapter{Literatūros apžvalga}

\ktusection{Pavojingos svetainės}

Didelė dalis interneto svetainių yra komerciniai projektai. Naudojami įvairūs pajamų šaltiniai.
Tai gali būti reklamos, vykdomi pardavimai, prenumeravimo principu veikiančios svetainės.
Šie būdai yra vartotojų matomi ir suprantami. Tačiau egzistuoja ir svetainės kurios pajamas
generuoja naudotojams kenkiančiais metodais. Vagiamos naudotojų tapatybės, banko sąskaitų, kreditinių kortelių
duomenys, užvaldomi kompiuteriai siekiant pasipelnyti. Tai yra pavojingos svetainės.

Pavojingos svetainės taip gali būti talpinamos ir ne savo noru.
Naudojantis spragomis programinėje įrangoje galima įterpti žalingą kodą į kitas svetaines,
taip pasinaudojant jų naudotojais ir jų pasitikėjimu svetaine.

Egzistuoja įvairūs būdai saugoti naudotojus nuo pavojingų svetainių \cite{trees}.
Vienas iš naudojamų variantų yra pavojingų svetainių blokavimas.
Naudotojui naršant internete svetainės analizuojamos pagal tam tikras savybes.
Svetainę galima analizuoti pagal įvairias metrikas \cite{tangled},
talpinančio serverio parametrus, raktažodžius adrese, asinchroniškai įkeliamus resursus.

Pavojingų svetainių aptikimo modulis yra pagrindinis tokios filtravimo sistemos komponentas.
Šis modulis sprendžia svetainių klasifikavimo problemą. Tam gali būti naudojami įvairūs mašininio mokymosi algoritmai.
Vienas iš puikiai tinkančių šiam panaudos atvejui \cite{trees} yra sprendimų medis.
Pagrindiniai jo privalumai yra didelis veikimo greitis, nesudėtingas derinimas - nėra daugybės hyperparametrų.

Modelio apmokymui naudojami egzistuojantys juodieji svetainių sąrašai.
Adresai sąrašuose yra skaldomi į žetonus kurie sudaryti iš žodžių, simbolių, skaičių.
Taip pat galima įtraukti ir svetainės metaduomenis - dydį, serverio duomenis, naudojamos programinės įrangos versijas.
Tačiau įprasti sprendimų medžiai nagrinėja vieną svetainę \cite{trees}.
Taip ignoruojama svarbiausia interneto savybė -- svetainės talpina nuorodas į kitas svetaines.
Šios nuorodos tiekia papildomą informaciją apie svetainę,
jos turinys bei rangas interneto grafe priklauso nuo svetainių kurios turi nuorodą į ją.
Šiuo principu yra pagrystas PageRank algoritmas \cite{pagerank} varantis "Google" paieškos variklį.
Laikant, kad svarbios svetainės turi nuorodas į kitas svarbias svetaines kyla mintis,
kad tai turi galioti ir pavojingoms svetainėms.
Šios svetainės siekia pritraukti žmones, dėl to turi būti optimizuotos SEO prasme.

žalingos svetainės gali būti analizuojamos pagal jų sąryšių grafą pagal \cite{webcop}



Interneto svetainių grafas yra biškį didelis.


\newpage