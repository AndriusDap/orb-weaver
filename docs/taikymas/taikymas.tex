\ktusection{Modelio naudojimas kenksmingų interneto tinklalapių identifikavimo programos kūrimui}

Apmokytas modelis yra paruoštas naudojimui, tačiau jis nėra itin patogus. Naudojamos programinės įrangos modelis gali
būti panaudojamas per komandinės eilutės sąsają, pateikiant duomenis mokymo metu naudotu formatu. Modelio grąžinamas
rezultatas taip pat nėra lengvai suvokiamas. Dėl to yra kuriama taikomoji programa kuri supaprastina modelio
panaudojimą realiose, praktinėse situacijose.

Kuriama sistema vadinama toliau vadinama \textbf{kenksmingų interneto tinklalapių identifikavimo programa}.

\begin{ktutable}{sasaja}{Kenksmingų interneto tinklalapių identifikavimo programos sąsaja}
    \begin{tabular}{|l|l|p{5cm}|p{5cm}|}
    \hline
        Kelias & Metodas & Parametrai & Rezultatas \\ \hline
        \texttt{/classify} & \texttt{GET} & \texttt{url} klasifikuojamo tinklalapio adresas & slankaus kablelio skaičius - klasifikavimo rezultatas \\ \hline
    \end{tabular}
\end{ktutable}

Kenksmingų interneto tinklalapių identifikavimo programa pritaikyta naudoti http protokolu, naudojama REST architektūra. Sąsaja pateikta \vref{tab:sasaja}
lentelėje. Tinklalpio kurį norima klasifikuoti adresas yra pateikiamas klasifikavimo programai, grąžinamas
rezultatas yra klasifikavimo modelio pateiktas įvertis.

Kenksmingų interneto tinklalapių identifikavimo programos architektūra pateikiama \ktufigref{images/architektura.png}

\ktufigure{images/architektura.png}{12 cm}{Kenksmingų interneto tinklalapių identifikavimo programos architektūra}

Kenksmingų interneto tinklalapių identifikavimo programa susideda iš dviejų komponentų:
\begin{enumerate}
\item sąsaja,
\item branduolys.
\end{enumerate}
Sąsaja yra komponentas kurio funkcija yra pateikti branduolio suteikiamą funkcionalumą naudotojams. Šiuo atveju
tai yra http protokolu veikianti sąsaja. Šis komponentas atsakingas už užklausos perdavimą branduolio komponentui,
branduolio komponente esančio klasifikatoriaus rezultato pateikimui naudotojui tinkamu formatu. Programinio
įrankio išeities kodas pateikiamas prieduose (2 priedas).

Branduolio komponente vykdomas modelio pritaikymas. Jį sudaro klasė \texttt{ClassifierController} bei duomenų
saugyklos kenksmingų ir nekenksmingų svetainių PageRank įverčiams.

Kenksmingų interneto tinklalapių identifikavimo programoje atskirti komponentai siekiant supaprastinti programos modifikacijas.. Atskiriant branduolio elementą kuriame yra
vykdoma verslo logika, matematiniai skaičiavimai sistema yra paruošiama greitam plėtimui, lengvam migravimui
tarp palaikomų ir naudojamų vartotojo sąsajos protokolų. Šiuo atveju vartotojo sąsaja nepriklauso nuo
branduolio, tik nuo klasės \texttt{ClassifierController} implementacijos.

Lyginant su alternatyvia sistema \cite{gapi} yra matomi kenksmingų interneto tinklalapių identifikavimo programos privalumai:
 \begin{enumerate}
    \item galima naudoti alternatyvi sąsaja, pakeitus sąsajos komponento implementaciją;
    \item klasifikavimo modelis gali būti keičiamas;
    \item nereikalauja interneto prieigos.
 \end{enumerate}

Pirminė programos versija yra kurta siekiant įrodyti jos veikimo koncepciją. Dėl to kai kurios sistemos vietos
yra optimizuotos programavimo laiko minimizavimo atžvilgiu. Plėtojant programinę įrangą vertėtų programą
skaidyti į kelis atskirtus komponentus, pagal naudojamų resursų tipą. PageRank įverčiai turi būti iškeliami
į duomenų bazę, mašininio mokymo modelis iškeliamas į atskirą vykdomąją aplikaciją kuri gali būti plečiamas
esant reikalui ar skaičiavimo resursų trūkumui.