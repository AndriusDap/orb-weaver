\ktusection{Modelio mokymas}

Modelio apmokymas vykdomas programinės įrangos paketu Vowpal Wabbit \cite{vw}.

\ktusubsection{Atramos vektorių mašinų modelio mokymas}

Naudojant atramos vektorių mašinų modelį sistema Vowpal Wabbit naudojami parametrai nurodyti \vref{tab:ksvm_params} lentelėje.

\begin{ktutable}{ksvm_params}{Atramos vektorių mašinų modelio hiperparametrai}
    \begin{tabular}{| l | l | p{7cm}|}
    \hline
        Komandinės eilutės parametras & Galimos vertės & Parametro reikšmė\\ \hline
        \texttt{passes} &  & Modelio mokymo iteracijų skaičius \\ \hline
        \texttt{loss\_function} &          & Mokymo metu naudojama nuostolio funkcija \\
                               & squared  & Modelio mokymo iteracijų skaičius \\
                               & hinge    & nuostolio funkcija $l(y) = \max(0, 1 - t * y)$ kur $y$ yra įvertis o $t = \pm 1$ tikroji klasė  \\
                               & logistic & logistinė nuostolio funkcija \\ \hline
        \texttt{l2} & & l2 parametrų reguliarizacija \\ \hline
        \texttt{bit\_precision} & & Bitų, naudojamų savybių lentelei, skaičius\\ \hline
        \texttt{kernel} & & Naudojama kernelio funkcija   \\
                       & poly & polinominis kernelis     \\
                       & linear & tiesinis kernelis      \\
                       & rbf & radialinės bazės kernelis \\ \hline
        \texttt{degree} & & polinominio kernelio laipsnis \\ \hline
    \end{tabular}
\end{ktutable}

Mokant atraminių vektorių mašinų modelį stebimas nuostolių funkcijos įvertis. Pagal jį vertinamas modelio mokymo sėkmingumas.

Modeliai pradėti mokyti nuo polinominio kernelio. Svarbūs parametrai derinant šį modelį yra polinominės funkcijos laipsnis bei parametrų reguliarizavimo koeficientas.

\begin{ktutable}{initial_train}{Pirminio polinominio kernelio modelio parametrai}
    \begin{tabular}{| l | c | c | c | c | c | c | }
    \hline
        Kernelio funkcija & polinominė & polinominė & polinominė & polinominė & polinominė & polinominė \\ \hline
        L2 reguliarizacija & $10^{-7}$ & $10^{-5}$ & 0.1 & $10^{-7}$ & $10^{-5}$ & 0.1 \\ \hline
        Kernelio funkcijos laipsnis & 5 & 5 & 5 & 3 & 3 & 3 \\ \hline
    \end{tabular}
\end{ktutable}

Mokant modelį naudojant parametrus nurodytus lentelėje \vref{tab:initial_train} pastebėta tendencija. Modelio kokybė mokymo metu negerėja. Tai atspindi šių parametrų nuostolio funkcijų grafikai \ktufigref{images/svm_sux.png}. Naudojant parametrų rinkinius iš lentelės \vref{tab:initial_train} modelis nekonverguoja.

\ktufigure{images/svm_sux.png}{14 cm}{Modelio kokybės tendencijos}

Modelio mokymas toliau tyrinėtas naudojant kitas branduolio funkcijas ir derinant jų parametrus. Modelio parametrų reguliarizacijos parametras L2 nekeistas.

\ktufigure{images/linear_loss.png}{12 cm}{Tiesinio kernelio modelio mokymas}
\ktufigure{images/polly_loss.png}{12 cm}{Polinominio kernelio modelio mokymas}
\ktufigure{images/rbf_loss.png}{12 cm}{Radialinės bazės funkcijos kernelio modelio mokymas}

Naudojant kitus kernelio tipus pastebėta ta pati tendencija, matoma \ktufigref{images/linear_loss.png}, \ktufigref{images/polly_loss.png}, \ktufigref{images/rbf_loss.png} grafikuose.