\ktusection{Modelio mokymas}

Modelio apmokymas vykdomas programinės įrangos paketu Vowpal Wabbit \cite{vw}.

\ktusubsection{Atramos vektorių mašinų modelio mokymas}

Naudojant atramos vektorių mašinų modelį sistema Vowpal Wabbit naudojami parametrai nurodyti \vref{tab:ksvm_params} lentelėje.

\begin{ktutable}{ksvm_params}{Atramos vektorių mašinų modelio hiperparametrai}
    \begin{tabular}{| l | l | p{7cm}|}
    \hline
        Komandinės eilutės parametras & Galimos vertės & Parametro reikšmė\\ \hline
        \texttt{passes} &  & Modelio mokymo iteracijų skaičius \\ \hline
        \texttt{loss\_function} &          & Mokymo metu naudojama nuostolio funkcija \\
                               & squared  & Modelio mokymo iteracijų skaičius \\
                               & hinge    & nuostolio funkcija $l(y) = \max(0, 1 - t * y)$ kur $y$ yra įvertis o $t = \pm 1$ tikroji klasė  \\
                               & logistic & logistinė nuostolio funkcija \\ \hline
        \texttt{l2} & & l2 parametrų reguliarizacija \\ \hline
        \texttt{bit\_precision} & & Bitų, naudojamų savybių lentelei, skaičius\\ \hline
        \texttt{kernel} & & Naudojama kernelio funkcija   \\
                       & poly & polinominis kernelis     \\
                       & linear & tiesinis kernelis      \\
                       & rbf & radialinės bazės kernelis \\ \hline
        \texttt{degree} & & polinominio kernelio laipsnis \\ \hline
    \end{tabular}
\end{ktutable}

Mokant atramos vektorių mašinų modelį stebimas nuostolių funkcijos įvertis. Pagal jį vertinamas modelio mokymo sėkmingumas.

Modeliai pradėti mokyti nuo polinominio kernelio. Svarbūs parametrai derinant šį modelį yra polinominės funkcijos laipsnis bei parametrų reguliarizavimo koeficientas.

\begin{ktutable}{initial_train}{Pirminio polinominio kernelio modelio parametrai}
    \begin{tabular}{| l | c | c | c | c | c | c | }
    \hline
        Kernelio funkcija & polinominė & polinominė & polinominė & polinominė & polinominė & polinominė \\ \hline
        L2 reguliarizacija & $10^{-7}$ & $10^{-5}$ & 0.1 & $10^{-7}$ & $10^{-5}$ & 0.1 \\ \hline
        Kernelio funkcijos laipsnis & 5 & 5 & 5 & 3 & 3 & 3 \\ \hline
    \end{tabular}
\end{ktutable}

Mokant modelį naudojant parametrus nurodytus lentelėje \vref{tab:initial_train} pastebėta tendencija. Modelio kokybė mokymo metu negerėja. Tai atspindi šių parametrų nuostolio funkcijų grafikai \ktufigref{images/svm_sux.png}. Naudojant parametrų rinkinius iš lentelės \vref{tab:initial_train} modelis nekonverguoja.

\ktufigure{images/svm_sux.png}{14 cm}{Modelio kokybės tendencijos}

Modelio mokymas toliau tyrinėtas naudojant kitas branduolio funkcijas ir derinant jų parametrus. Modelio parametrų reguliarizacijos parametras L2 nekeistas.

\ktufigure{images/linear_loss.png}{12 cm}{Tiesinio kernelio modelio mokymas}
\ktufigure{images/polly_loss.png}{12 cm}{Polinominio kernelio modelio mokymas}
\ktufigure{images/rbf_loss.png}{12 cm}{Radialinės bazės funkcijos kernelio modelio mokymas}

Naudojant kitus kernelio tipus pastebėta ta pati tendencija, matoma \ktufigref{images/linear_loss.png}, \ktufigref{images/polly_loss.png}, \ktufigref{images/rbf_loss.png} grafikuose.

\ktusubsection{Tiesinio modelio mokymas}

Vowpal Wabbit sistema visiems modeliams naudoja vienodą duomenų formatą bei paanšius parametrus. Tai leidžia lengvai bandyti įvairius modelius. Dėl to po prasto atramos vektorių mašinų modelių pasirinktas vienas iš paprasčiausių modelių, kuriuo tiėktina pasiekti gerus rezultatus, nenaudojant didelio resursų kiekio modelio formavimui.

\begin{ktutable}{linear_params}{Tiesinio modelio hiperparametrai}
    \begin{tabular}{| l | l | p{7cm}|}
    \hline
        Komandinės eilutės parametras & Galimos vertės & Parametro reikšmė\\ \hline
        \texttt{passes} &  & Modelio mokymo iteracijų skaičius \\ \hline
        \texttt{loss\_function} &          & Mokymo metu naudojama nuostolio funkcija \\
                               & squared  & Modelio mokymo iteracijų skaičius \\
                               & hinge    & nuostolio funkcija $l(y) = \max(0, 1 - t * y)$ kur $y$ yra įvertis o $t = \pm 1$ tikroji klasė  \\
                               & logistic & logistinė nuostolio funkcija \\ \hline
        \texttt{l2} & & l2 parametrų reguliarizacija \\ \hline
        \texttt{bit\_precision} & & Bitų, naudojamų savybių lentelei, skaičius\\ \hline
    \end{tabular}
\end{ktutable}

Mokant tiesinį modelį naudojamas mažesnis sąvybių lentelės dydis nurodomas naudojant \texttt{bit\_precision} parametrą. Tai leidžia optimialiau atlikti mokymą sumažinant naudojamą atminties kiekį. Naudojami $27$ bitai.

Pirminiai modelio mokymo rezultatai pateikiami \ktufigref{images/linearModelLoss.png}. Šiame grafike matoma, kad nuostolio funkcijos rezultatai yra žymiai geresni nei mokant atramos veotkrių mašinų modelius. Mokomas modelis konverguoja, pasiekiama gana pastovi nuostolio funkcijos reikšmė, kuri yra nedidelė, artima nuliui.

\ktufigure{images/linearModelLoss.png}{14 cm}{Tiesinio modelio mokymas}

Apmokius modelį vertinamas jo jautrumas, tikslumas, specifiškumas, AUC įvertis. Jie pateikiami lentelėje \vref{tab:firstLinearModel}. Parametrų L2 reguliarizavimui naudota vertė $10^{-6}$.
Gautas modelis itin gerai prognozuoja testavimo rinkinio duomenis. Suformuotas tiesinis modelis naudoja
$145832$ matmenų.

\begin{ktutable}{firstLinearModel}{Pirminio tiesinio modelio rezultatai}
    \begin{tabular}{| l | l |}
    \hline
       Matas & Reikšmė \\ \hline
               Tikslumas & 0.97552 \\ \hline
               Jautrumas & 0.92588 \\ \hline
               Specifiškumas & 0.98325 \\ \hline
               AUC & 0.95456 \\ \hline
    \end{tabular}
\end{ktutable}

\ktufigure{images/roc.png}{12 cm}{Tiesinio modelio ROC kreivė}

Lyginant šio modelio rezultatus pateiktus \cite{comp} straipsnyje šis tiesinis modelis pralenkia atramos vektorių mašinų klasifikatorių, kuris šiame strapsnyje buvo išskiriamas kaip geriausias klasifikatorius.

\ktusubsection{Sprendimų medžio modelio formavimas}

Paskutinis naudojamas modelis yra sprendimų medis. Modelio naudojami parametrai pateikiami lentelėje \vref{tab:medisParams}.


\begin{ktutable}{medisParams}{Sprendimų medžio modelio hiperparametrai}
    \begin{tabular}{| l | l | p{7cm}|}
    \hline
        Komandinės eilutės parametras & Galimos vertės & Parametro reikšmė\\ \hline
        \texttt{passes} &  & Modelio mokymo iteracijų skaičius \\ \hline
        \texttt{loss\_function} &          & Mokymo metu naudojama nuostolio funkcija \\
                               & squared  & Modelio mokymo iteracijų skaičius \\
                               & hinge    & nuostolio funkcija $l(y) = \max(0, 1 - t * y)$ kur $y$ yra įvertis o $t = \pm 1$ tikroji klasė  \\
                               & logistic & logistinė nuostolio funkcija \\ \hline
        \texttt{l2} & & l2 parametrų reguliarizacija \\ \hline
        \texttt{bit\_precision} & & Bitų, naudojamų savybių lentelei, skaičius\\ \hline
        \texttt{log\_multi} & & Siekiamas sprendimų medžio kompeksiškumas laiko atžvilgiu, $O(log(n))$ kur $n$ yra šio parametro vertė\\ \hline
    \end{tabular}
\end{ktutable}

Sprendimų medžio modelio mokymas vyksta sėkmingai, stebint nuostolio funkcijos vertes \ktufigref{images/treeLoss.png}.

\ktufigure{images/treeLoss.png}{12 cm}{Sprendimų medžio modelio nuostolio funkcija}

Nepaisant gero nuostolio funkcijos rezultato šis modelis persimokė su visomis naudotomis hiperparametrų vertėmis. Šis modelis visada prognozuoja tik vieną iš reikšmių, tai yra klasifikacija nevykdoma, modelis visada grąžina kad analizuojaams vektorius priklauso klasei 1 (svetainė nežalinga).

\ktusubsection{Modelių mokymo rezultatai}

Kokybiško atramos vektorių mašinų modelio bei sprendimų medžio apmokyti nepavyko. Tikėtina, kad tai lemia naudojamo duomenų tipo specifika. Duomenys turi itin daug matmenų - kiekvienas naujas žodis, simbolių grupė yra koduojama kaip naujas matmuo. Tačiau suformuotas itin tikslus tiesinis modelis. Lyginant tiesinio modelio rezultatus su pateiktais įvairių modelių rezultatais straipsnyje \cite{comp}, kurie pateikiami lentelėje \vref{tab:compTable} visą mokymą galima laikyti pasisekusiu. Nors apmokytas tik vienas modelis iš trijų jis yra itin tikslus ir lenkia atramos vektorių mašinų modelį, kuris buvo įvertintas kaip geriausias.

\begin{ktutable}{compTable}{Pirminio tiesinio modelio rezultatai}
    \begin{tabular}{|l|p{3cm}|p{3cm}|p{3cm}|p{3cm}|}
    \hline
                             & Tiesinis modelis  &  K Artimiausių kaimynų modelis  &  Radialinės bazės funkcijos kernelio atramos vektorių mašinos  &  Tiesinio kernelio atramos vektorių mašinos               \\ \hline
               Tikslumas     & 0.97552           & 0.91   & 0.97  &  0.92         \\ \hline
    \end{tabular}
\end{ktutable}

Šis modelis toliau naudojamas klasifikavimo įrankio kūrime.