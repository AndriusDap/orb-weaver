\ktusection{Modelio mokymas}

Modelių mokymas buvo vykdomas programinės įrangos paketu Vowpal Wabbit \cite{vw}.

\ktusubsection{Atraminių vektorių klasifikatoriaus mokymas}

Naudojant atraminių vektorių klasifikatorių sistemoje Vowpal Wabbit naudojami parametrai nurodyti \vref{tab:ksvm_params} lentelėje.

\begin{ktutable}{ksvm_params}{Atraminių vektorių klasifikatoriaus hiperparametrai}
    \begin{tabular}{| l | l | p{7cm}|}
    \hline
        Komandinės eilutės parametras & Galimos vertės & Parametro reikšmė\\ \hline
        \texttt{passes} &  & Modelio mokymo iteracijų skaičius \\ \hline
        \texttt{loss\_function} &          & Mokymo metu naudojama nuostolio funkcija: \\
                               & hinge    & nuostolio funkcija $l(y) = \max(0, 1 - t * y)$ kur $y$ yra įvertis o $t = \pm 1$ tikroji klasė  \\
                               & logistic & logistinė nuostolio funkcija \\ \hline
        \texttt{l2} & & l2 parametrų reguliarizacija \\ \hline
        \texttt{bit\_precision} & & Bitų, naudojamų savybių lentelei, skaičius\\ \hline
        \texttt{kernel} & & Naudojama branduolio funkcija:   \\
                       & poly & polinominis branduolys     \\
                       & linear & tiesinis branduolys      \\
                       & rbf & radialinės bazės branduolys \\ \hline
        \texttt{degree} & & polinominio branduolio laipsnis \\ \hline
    \end{tabular}
\end{ktutable}

Mokant atraminių vektorių klasifikatorių stebimas nuostolių funkcijos įvertis. Pagal jį
vertinamas modelio mokymo kokybė.

Modeliai pradėti mokyti nuo polinominio branduolio. Svarbūs parametrai derinant šį modelį yra polinominės
funkcijos laipsnis bei parametrų reguliarizavimo koeficientas.

\begin{ktutable}{initial_train}{Pirminio polinominio branduolio modelio parametrai}
    \begin{tabular}{| l | c | c | c | c | c | c | }
    \hline
        L2 reguliarizacija & $10^{-7}$ & $10^{-5}$ & 0,1 & $10^{-7}$ & $10^{-5}$ & 0,1 \\ \hline
        Branduolio funkcijos laipsnis & 5 & 5 & 5 & 3 & 3 & 3 \\ \hline
    \end{tabular}
\end{ktutable}

Modelio mokymas toliau tyrinėtas naudojant kitas branduolio funkcijas ir derinant jų parametrus.
Modelio parametrų reguliarizacijos parametras L2 nekeistas.

\ktufigure{images/linear_loss.png}{12 cm}{Tiesinio branduolio modelio nuostolių ir tinklalapių skaičiaus naudoto mokymui priklausomybė}

\ktusubsection{Logistinės regresijos modelio mokymas}

Vowpal Wabbit sistema visiems modeliams naudoja vienodą duomenų formatą bei panašius parametrus.
Tai leidžia lengvai bandyti įvairius modelius. Dėl to po prastų atraminių vektorių klasifikatoriaus rezultatų
pasirinktas vienas iš paprasčiausių modelių, kuriuo tikėtina pasiekti gerus rezultatus, nenaudojant
didelio resursų kiekio modelio formavimui.

\begin{ktutable}{linear_params}{Logistinės regresijos modelio hiperparametrai}
    \begin{tabular}{| l | l | p{7cm}|}
    \hline
        Komandinės eilutės parametras & Galimos vertės & Parametro reikšmė\\ \hline
        \texttt{passes} &  & Modelio mokymo iteracijų skaičius \\ \hline
        \texttt{loss\_function} &          & Mokymo metu naudojama nuostolio funkcija \\
                               & hinge    & nuostolio funkcija $l(y) = \max(0, 1 - t * y)$ kur $y$ yra įvertis o $t = \pm 1$ tikroji klasė  \\
                               & logistic & logistinė nuostolio funkcija \\ \hline
        \texttt{l2} & & l2 parametrų reguliarizacija \\ \hline
        \texttt{bit\_precision} & & Bitų, naudojamų savybių lentelei, skaičius\\ \hline
    \end{tabular}
\end{ktutable}

Mokant logistinės regresijos modelį naudojamas mažesnis savybių lentelės dydis nurodomas naudojant \texttt{bit\_precision}
 parametrą. Tai leidžia optimaliau atlikti mokymą sumažinant naudojamą atminties kiekį. Naudojami $27$ bitai.

Modelio mokymo nuostoliai pateikiami \ktufigref{images/linearModelLoss.png}. Šiame grafike
matoma, kad nuostolio funkcijos rezultatai yra žymiai geresni nei mokant atraminių vektorių klasifikatorius. Mokomas modelis konverguoja, pasiekiama gana pastovi nuostolio funkcijos reikšmė, artima nuliui.

\ktufigure{images/linearModelLoss.png}{14 cm}{Logistinės regresijos modelio nuostolių priklausomybė nuo mokymui naudojamų tinklalapių skaičiaus}


Gautas modelis itin gerai prognozuoja testavimo rinkinio duomenis. Suformuotas logistinės regresijos modelis naudoja
$145832$ matmenų.

\ktufigure{images/roc.png}{12 cm}{Logistinės regresijos modelio ROC kreivė}

Lyginant šio modelio rezultatus pateiktus \cite{comp} straipsnyje šis logistinės regresijos modelis pralenkia
atraminių vektorių klasifikatorių, kuris šiame straipsnyje buvo išskiriamas kaip geriausias klasifikatorius.

\ktusubsection{Sprendimų medžio modelio formavimas}

\ktufigure{images/treeLoss.png}{12 cm}{Sprendimų medžio modelio nuostolių priklausomybė nuo mokymui naudojamų tinklalapių skaičiaus}

Sprendimų medžio modelio mokymo parametrai pateikiami lentelėje \vref{tab:medisParams}.

\begin{ktutable}{medisParams}{Sprendimų medžio modelio hiperparametrai}
    \begin{tabular}{| l | l | p{7cm}|}
    \hline
        Komandinės eilutės parametras & Galimos vertės & Parametro reikšmė\\ \hline
        \texttt{passes} &  & Modelio mokymo iteracijų skaičius \\ \hline
        \texttt{loss\_function} &          & Mokymo metu naudojama nuostolio funkcija \\
                               & hinge    & nuostolio funkcija $l(y) = \max(0, 1 - t * y)$ kur $y$ yra įvertis o $t = \pm 1$ tikroji klasė  \\
                               & logistic & logistinė nuostolio funkcija \\ \hline
        \texttt{l2} & & l2 parametrų reguliarizacija \\ \hline
        \texttt{bit\_precision} & & Bitų, naudojamų savybių lentelei, skaičius\\ \hline
        \texttt{log\_multi} & & Siekiamas sprendimų medžio kompleksiškumas laiko atžvilgiu, $O(log(n))$ kur $n$ yra šio parametro vertė\\ \hline
    \end{tabular}
\end{ktutable}

Sprendimų medžio modelio mokymas vyksta sėkmingai, stebint nuostolio funkcijos vertes paveiksle \ktufigref{images/treeLoss.png}.

Nepaisant gero nuostolio funkcijos rezultato šis modelis persimokė su visomis naudotomis
hiperparametrų vertėmis. Šis modelis visada prognozuoja tik vieną iš reikšmių, tai yra
klasifikacija nevykdoma, modelis visada grąžina, kad analizuojamas vektorius priklauso klasei 1 (tinklalapis kenksmingas).

\ktusubsection{Modelių mokymo rezultatai}


\begin{ktutable}{firstLinearModel}{Modelių kokybės įverčiai}
    \begin{tabular}{| l | p{4cm} | p{4cm} | p{4cm} |}
    \hline
               Matas & Logistinės regresijos klasifikatorius & Atraminių vektorių klasifikatorius & Sprendimų medžio klasifikatorius \\\hline
               Tikslumas & 0,97552 & 0,13473 & 0,13473 \\ \hline
               Jautrumas & 0,92588 & 1,00000 & 1,00000 \\ \hline
               Specifiškumas & 0,98325 & 0 & 0 \\ \hline
               AUC & 0.95456 & 0,5 & 0,5 \\ \hline
    \end{tabular}
\end{ktutable}

Kokybiško atraminių vektorių klasifikatoriaus bei sprendimų medžio apmokyti nepavyko.
Tikėtina, kad tai lemia naudojamo duomenų tipo specifika. Visi klasifikavimo kokybės įverčiai pateikiami lentelėje \vref{tab:firstLinearModel}. Duomenys turi itin daug
matmenų - kiekvienas naujas žodis, simbolių grupė yra koduojama kaip naujas matmuo.
Tačiau suformuotas itin tikslus logistinės regresijos modelis. Lyginant logistinės regresijos modelio rezultatus
su pateiktais įvairių modelių rezultatais straipsnyje \cite{comp}, kurie pateikiami
lentelėje \vref{tab:compTable} visą mokymą galima laikyti sėkmingu. Nors apmokytas
tik vienas modelis iš trijų jis yra itin tikslus ir lenkia atraminių vektorių mašinų klasifikatorių, kuris buvo įvertintas kaip geriausias.

\begin{ktutable}{compTable}{Logistinės regresijos modelio kokybės metrikos lyginamos su \cite{comp}}
    \begin{tabular}{|l|p{3cm}|p{3cm}|p{3cm}|p{3cm}|}
    \hline
                             & Logistinės regresijos modelis  &  K-artimiausių kaimynų modelis  &  Radialinės bazės funkcijos branduolio atraminių vektorių klasifikatorius  &  Tiesinio branduolio atraminių vektorių klasifikatorius \\ \hline
               Tikslumas     & 0.97552           & 0.91   & 0.97  &  0.92         \\ \hline
    \end{tabular}
\end{ktutable}

Šis modelis toliau naudojamas klasifikavimo įrankio kūrime.
