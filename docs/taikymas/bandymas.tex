\ktusection{Praktinis kenksmingų interneto tinklalapių identifikavimo programos taikymas}

Sukurtas įrankis bandytas panaudoti naudojant kelias nekenksmingas interneto svetaines. Dalis iš bandytų
tinklalapių pateikiama lentelėje \vref{tab:bandymas}.

\begin{ktutable}{bandymas}{Kenksmingų interneto tinklalapių identifikavimo programos bandymo rezultatai}
    \begin{tabular}{|p{5 cm}|p{5 cm}|p{3 cm}|}
    \hline
    Tinklalapis & Tinklalapio adresas & Kenksmingas \\ \hline
    KTU titulinis tinklalapis & http://ktu.edu/ & Ne \\ \hline
    Wikipedia straipsnis apie dramblius & https://lt.wikipedia.org/ wiki/Drambliai & Ne \\ \hline
    Socialinis tinklas Facebook & https://facebook.com & Taip \\ \hline
    Google paieškos variklis & https://google.com & Taip \\ \hline
    15 min laikraštis & https://15min.lt & Taip \\ \hline
    \end{tabular}
\end{ktutable}

Toks kenksmingų interneto tinklalapių identifikavimo programos bandymas parodo trūkumų. Tam tikri tinklalapiai kaip Google, 15min kurie nėra kenksmingi,
yra klasifikuojami netiesingai. Tačiau validavimo imties klasifikavimo kokybės metrikos rodo gerus rezultatus - \textbf{tikslumas 0.976, specifiškumas 0.982, jautrumas 0.932}.

Programos trūkumas yra tai, kad iš tiesinės regresijos modelio negalima paprastai gauti priežasčių, dėl ko tinklalapis buvo klasifikuotas kaip kenksmingas ar nekenksmingas. Tačiau kenksmingų interneto tinklalapių identifikaivmo programa turi ir privalumų. Kenksmingų tinklalapių klasifikavimas vykdomas pagal mokymo duomenų rinkinį, tai leidžia pakeitus kenksmingų tinklalapių sąrašą identifikuoti kitos klasės tinklalapius, pavyzdžiui pagal jų tematiką. Taip pat sistemos veikimas yra aiškus - komercinės sistemos nepateikia tikslaus klasifikavimo metodo. Šią programą galima lengvai modifikuoti, ar keičiant sistemos sąsają ar klasifikavimo modelį. Taikant programą praktikoje būtina atsižvelgti į šiuos trūkumus bei privalumus.