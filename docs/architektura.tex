\ktuchapter{Sistemos architektūra}

Kuriamos programinės įrangos tikslas -- atpažinti ar svetainė yra kenkėjiška.

Sistemai keliami tam tikri nefunkciniai reikalavimai. Galimybė plėsti sistemai skirtus kompiuterinius resursus, optimaliai
panaudojant turimus resursus. Privaloma leisti mažinti infrastruktūros kaštus jei sistemos apkrova yra nedidelė, bei
skirti papildomus skaičiavimo resursus esant didelei apkrovai.

Duotus nefunkcinius reikalavimus leidžia išpildyti mikroservisais paremta architektūra. Komponentai yra kuriami kaip
atskiros sistemos kurios pateikiamos Docker sistemos konteineriuose. Komunikacijai tarp konteinerių naudojamas HTTPS
protokolas, bendravimas paremtas REST principais.

Docker konteineriai naudojajmi siekiant sistemos komponentus padaryti nepriklausomus nuo išorinių resursų. Visos
reikalingos bibliotekos, sistemos yra sudiegiamos kartu su konteineriu. Tai itin palengvina horizontalų sistemos
plėtojimą padidėjus apkrovai. Vieno tipo konteineriai yra sudiegiami taip gaunant kelias skirtingas tarnybas
atliekančias skirtingas funkcijas. Kurios nors tarnybos apkrovai didėjant skiriama daugiau serverinės resursų šiai
tarnybai -- kuriamos naujos sistemos naudojant egzistuojančius konteinerius.

Siekiant įgyvendinti horizontalų plėtimąsi tarnybos turi veikti naudojant skirtingus servisų kiekius. Tai gali būti
įgyvendinama naudojant kelis skirtingus metodus. Naudojant žinučių eiles užklausos tarnyboms yra formuojamos ir dedamos
į eilę. Tarnybos posistemės iš eilės ima žinutes, jas apdoroja ir siunčia atsakymus. Alternatyvus metodas yra HTTPS
protokolo naudojimas su REST tipo paslaugomis. Naudojant šią metodiką naudojami įprasto apkrovos balansavimo įrankiai.
Tai taip pat leidžia lengviau naudoti atskirus modulius vartotojus pasiekiančiose aplikacijose, su sistema nesusijusiose
programose. Taip yra dėdl didesnio REST paplitimo, lyginant su žinučių eilėmis, paprastesnio sistemos diegimo.

Sistema sudaroma iš kelių komponentų. Interneto grafo analizei pasirenkamas skenavimas iš apačios aukštyn
remiantis \cite{webcop} idėjomis. Svetainės esančios kaimynystėje yra vertinamos naudojant PageRank algoritmo
principus \cite{pagerank} siekiant aprėpti daugiau nei vieną kaimyną. Žalingų svetainių atpažinimui naudojamas
mašininio mokymosi modelis \cite{trees}.

\ktusection{Svetainių klasifikavimo modulis}
\ktusection{Gretimų svetainių aptikimo modulis}
\ktusection{Subgrafo vertinimo modulis}
\ktusection{Spartinančioji atmintinė}
\newpage