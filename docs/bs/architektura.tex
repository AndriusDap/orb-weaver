\ktuchapter{Sistemos architektūra}

Kuriamos programinės įrangos tikslas -- atpažinti ar svetainė yra kenkėjiška.

Sistemai keliami tam tikri nefunkciniai reikalavimai. Galimybė plėsti sistemai skirtus kompiuterinius resursus, optimaliai
panaudojant turimus resursus. Privaloma leisti mažinti infrastruktūros kaštus jei sistemos apkrova yra nedidelė, bei
skirti papildomus skaičiavimo resursus esant didelei apkrovai.

Duotus nefunkcinius reikalavimus leidžia išpildyti mikroservisais paremta architektūra. Komponentai yra kuriami kaip
atskiros sistemos kurios pateikiamos Docker sistemos konteineriuose. Komunikacijai tarp konteinerių naudojamas HTTPS
protokolas, bendravimas paremtas REST principais.

Docker konteineriai naudojajmi siekiant sistemos komponentus padaryti nepriklausomus nuo išorinių resursų. Visos
reikalingos bibliotekos, sistemos yra sudiegiamos kartu su konteineriu. Tai itin palengvina horizontalų sistemos
plėtojimą padidėjus apkrovai. Vieno tipo konteineriai yra sudiegiami taip gaunant kelias skirtingas tarnybas
atliekančias skirtingas funkcijas. Kurios nors tarnybos apkrovai didėjant skiriama daugiau serverinės resursų šiai
tarnybai -- kuriamos naujos sistemos naudojant egzistuojančius konteinerius.

Siekiant įgyvendinti horizontalų plėtimąsi tarnybos turi veikti naudojant skirtingus servisų kiekius. Tai gali būti
įgyvendinama naudojant kelis skirtingus metodus. Naudojant žinučių eiles užklausos tarnyboms yra formuojamos ir dedamos
į eilę. Tarnybos posistemės iš eilės ima žinutes, jas apdoroja ir siunčia atsakymus. Alternatyvus metodas yra HTTPS
protokolo naudojimas su REST tipo paslaugomis. Naudojant šią metodiką naudojami įprasto apkrovos balansavimo įrankiai.
Tai taip pat leidžia lengviau naudoti atskirus modulius vartotojus pasiekiančiose aplikacijose, su sistema nesusijusiose
programose. Taip yra dėdl didesnio REST paplitimo, lyginant su žinučių eilėmis, paprastesnio sistemos diegimo.

Sistema sudaroma iš kelių komponentų. Interneto grafo analizei pasirenkamas skenavimas iš apačios aukštyn
remiantis \cite{webcop} idėjomis. Svetainės esančios kaimynystėje yra vertinamos naudojant PageRank algoritmo
principus \cite{pagerank} siekiant aprėpti daugiau nei vieną kaimyną. Žalingų svetainių atpažinimui naudojamas
mašininio mokymosi modelis \cite{trees}.

\ktusection{Svetainių klasifikavimo modulis}

Svetainių klasifikavimo modulio paskirtis - atsakyti ar svetainė yra kenksminga ar ji yra saugi naudotojams.
Modulį sudaro matematinis modelis, duomenų paruošimo komponentas, sąsaja skirta bendradarbiauti su kitomis tarnybomis.

Kenkėjiškoms svetainėms atpažinti naudojamas sprendimų medis. Šis algoritmas neveikai ant įprastų, adresus
atvaizduojančių, simbolių eilučių. Dėl to iš svetainių adresų yra suformuojami įvairūs parametrai. Šiuos parametrus
sudaro svetainių adresų ilgis, skyrybos simbolių skaičius adrese. Adresai taip pat suskaidomi į simbolius bei naudojamus
žodžius. Apmokant modelį žodžiai yra svetainės sąvybės kurios gali turėti dvi reikšmes - 0 arba 1, priklausomai nuo to
ar žodis egzistuoja ar neegzistuoja svetainėje. Modeliui apmokyti naudojamas VowpalWabbit programinės įrangos paketas.
Ši programa parinkta dėl galimybės naudoti skirtingas tekstines adreso sąvybes nesudarant viso galimo žodyno. Tai yra
įgyvendinama skaičiuojant maišos kodus visiems atskiriems žodžiams. Naudojant tokį žodyno vengimo būdą svarbu parinkti
pakankamą atminties kiekį maišos lentelės formavimui. Išskyrus nepakankamą atminties kiekį yra rizikuojama naudoti per
trumpus maišos kodus, kas sukels jų susikirtimus. Maišos kodų sutapimai lemtų mažą modelio tikslumą.

Modeliui apmokyti naudojamas žinomų kenksmingų svetainių sąrašas, suformuotas iš įvairių blokavimo sąrašų, bei
atsitiktinai parinktos svetainės iš interneto grafo. Siekiant aptikti modelio persimokymą naudojama kryžminė
validacija.

Viena iš naudingų VowpalWabbit programinės įrangos sąvybių yra galimybė naudoti tą patį duomenų rinkinį su ksirtingais
mašininio mokymosi modeliais. Tai reikšia, kad derinant modelio hyperparametrus, siekiant išgauti geriausią rezultatą
galima keisti ne tik įprastus parametrus bet ir patį modelį. Ttai leidžia išmėginti ne tik sprendimų medžius, bet ir
atsitiktinio miško metodą, neuroninius tinklus. Visi šie modeliai ir jų hyperparametrai privalo būti išbandyti,
siekiant surasti optimaliausią modelį turimam duomenų rinkiniui.

Apmokius modelį yra įvertinamas jo tikslumas, tikrinama, ar modelis nėra permokytas. Jei modelis buvo sukurtas sėkmingai,
konstruojamas svetainių klasifikavimo modulio konteineris kuris bus naudojamas visoje sistemoje.

\ktusection{Subgrafų radimo modulis}

Gretimų svetainių aptikimo modulio tikslas -- surasti analizuojamos svetainės kaimynus. Tai atliekama vykdant paiešką
iš apačios viršun per intereneto svetainių grafą. Naudojamas interneto grafas kelia sunkumų dėl jo dydžio. Naudojamas
duomenų rinkinys turi 1.8 milijardo įrašų. Tai sudaro apie 200 gigabaitų informacijos. Visa ši informacija turi būti
pasiekiama realiu laiku, dėl to ji negali būti suarchyvuota ar iškelta į kitą serverį.
Šią problemą sprendžia WebGraph programinis paketas. Šis paketas leidžia saugoti tik grafo viršūnių indeksus bei ryšius
 tarp jų. Tai leidžia sutalpinti grafo informaciją į 30 gigabaitų dydžio failą. Tačiau šiame grafe nėra saugomi
 svetainių adresai, tik jų indeksai. Kadangi svarbu žinoti žalingų svetainių adresus yra suformuojamos dvi duomenų
 bazės. Viena iš jų leidžia pagal svetainės adresą gauti jos indeksą, kita atvirkščiai, pagal indeksą gauti svetainės
 adresą. Šios tarpinės duomenų bazės telpa į 30 gigabaitų vietą diske.

\ktusection{Subgrafo vertinimo modulis}
\newpage